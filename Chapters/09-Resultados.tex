\chapter[Resultados y Análisis]{Resultados y Análisis}
\label{chap:resultados}

\insertminitoc
\parindent0pt


\section{Presentación de Resultados}
Los resultados que se lograron después del análisis estructural de la red, la elaboración del grafo vial y la realización de las simulaciones en \Acrshort{sumo} en varios 
escenarios (congestión, condición base y reemplazo de rotonda por semáforo) se presentan en esta sección.  Se muestran los resultados a través de gráficos de tendencia y 
métricas cuantitativas, que posibilitan examinar el funcionamiento general del sistema de movilidad.

\begin{figure}[H]
    \centering
    \includegraphics[width=1\linewidth]{Figures/Resultados/pronostico_parque_vehicular_hn.png}
    \caption[Pronóstico del Parque Vehicular en Honduras]{Pronóstico del Parque Vehicular en Honduras. Fuente: Elaboración propia.}
    \label{fig:figure-pronostico-parque-vehicular-hn}
\end{figure}

Esta gráfica muestra un modelo de regresión lineal que se utiliza para analizar el desarrollo del parque vehicular en Comayagua. Se basa en datos históricos del Instituto 
Nacional de Estadística (INE) entre 2013 y 2021, los cuales indican una tendencia de crecimiento sostenido y con una correlación significativa.  Para 2030, se prevé que la 
flota de vehículos excederá las 225,000 unidades, lo que representa un aumento cercano al 30\% en comparación con la estimación del parque actual. Este patrón lineal revela 
una etapa de expansión de la motorización, que no muestra señales de saturación desordenada y prevé una presión creciente sobre la infraestructura vial y el planeamiento 
urbano regional en la próxima década.

\begin{figure}[H]
    \centering
    \includegraphics[width=1\linewidth]{Figures/Resultados/graph_centrality.pdf}
    \caption[Análisis Topológico y de Vulnerabilidad: Centralidad de Intermediación de la Red Vial de Comayagua]{Análisis Topológico y de Vulnerabilidad: Centralidad de Intermediación de la Red Vial de Comayagua. Fuente: Elaboración propia.}
    \label{fig:figure-graph-centrality}
\end{figure}

Esta representación topológica, probablemente producida a través de análisis de grafos, muestra la centralidad de intermediación en la red vial de Comayagua. Los nodos 
(intersecciones) y las aristas (calles) están codificados cromáticamente para señalar su jerarquía funcional; el núcleo central presenta una concentración de tonos claros 
(amarillo y cian), lo que marca los únicos puntos de falla en la red. Esto indica que un alto porcentaje de las rutas más cortas tienen que pasar por un número limitado de 
segmentos y nudos. Por lo tanto, si estas áreas críticas se bloquean o inhabilitan, la conectividad total del sistema se vería gravemente afectada, provocando una ineficiencia 
no lineal en los tiempos de viaje y evidenciando una escasa redundancia estructural frente a cualquier alteración vial.

Habiendo identificado la intersección como nodo crítico dentro de la red vial, se procedió a la modelización micro-simulada de tráfico para evaluar su impacto operativo y 
cuantificar su desempeño en condiciones dinámicas. De esta aproximación metodológica se extrajeron indicadores clave de rendimiento (KPIs), permitiendo la medición precisa de 
las condiciones de flujo vehicular a través de la velocidad media, la pérdida de tiempo y paradas por vehículo.

\begin{figure}[H]
    \centering
    \includegraphics[width=1\linewidth]{Figures/Resultados/comparativa_barras_individual_interval_meanSpeedWithin_estadio.pdf}
    \caption[Simulación de Velocidad Media bajo Escenarios de Crecimiento Vehicular]{Simulación de Velocidad Media bajo Escenarios de Crecimiento Vehicular. Fuente: Elaboración propia.}
    \label{fig:figure-velocidad-media}
\end{figure}

El gráfico de barras presenta una comparativa de la velocidad media (\textbf{$\text{m/s}$}) de circulación para una intersección específica de Comayagua bajo tres escenarios 
proyectados, sirviendo como un indicador clave de rendimiento operacional de la red vial. Los "Datos año 2025" (escenario actual) establecen la línea base con 
$\mathbf{3.21\ \text{m/s}}$; la proyección al "Datos año 2030" (escenario de crecimiento lineal) muestra una ligera disminución a $\mathbf{3.13\ \text{m/s}}$, reflejando 
una moderada afectación por el incremento vehicular predicho. No obstante, el escenario hipotético de estrés máximo ("Datos año 20XX"), en el que se duplica el parque 
vehicular, provoca una caída notable de la velocidad media a $\mathbf{3.00\ \text{m/s}}$; esta diferencia progresiva confirma la sensibilidad de la intersección al 
volumen de tráfico, sugiriendo que, si bien la infraestructura puede absorber el crecimiento proyectado a 2030 con un impacto mínimo, el incremento exponencial del parque 
vehicular resultaría en una degradación significativa de la fluidez operacional.

\clearpage

\begin{figure}[H]
    \centering
    \includegraphics[width=1\linewidth]{Figures/Resultados/comparativa_barras_individual_interval_meanTimeLossWithin_estadio.pdf}
    \caption[Simulación de Pérdida de Tiempo bajo Escenarios de Crecimiento Vehicular]{Simulación de Pérdida de Tiempo bajo Escenarios de Crecimiento Vehicular. Fuente: Elaboración propia.}
    \label{fig:figure-perdida-tiempo}
\end{figure}

El gráfico de barras ilustra la pérdida de tiempo promedio (\textbf{en segundos}) experimentada por los vehículos al transitar por la intersección crítica, un indicador clave del nivel 
de congestión y fricción en el flujo, bajo los tres escenarios de crecimiento vehicular planteados. En el escenario actual ("Datos año 2025"), la pérdida de tiempo es de 
$\mathbf{23.53\ \text{s}}$. La proyección hacia el "Datos año 2030" muestra un aumento significativo a $\mathbf{25.12\ \text{s}}$, confirmando que la expansión vehicular 
proyectada ejercerá una presión perceptible sobre el sistema, resultando en una mayor ineficiencia temporal. Finalmente, el escenario de estrés máximo ("Datos año 20XX"), que 
simula la duplicación del parque vehicular, induce la mayor pérdida de tiempo, alcanzando $\mathbf{25.45\ \text{s}}$; este incremento marginal respecto a 2030 
($25.45\ \text{s}$ vs $25.12\ \text{s}$) sugiere que, si bien la congestión se intensifica con el crecimiento, la intersección podría estar cercana a su punto de saturación, 
donde las penalizaciones temporales adicionales, aunque existentes, se vuelven menos dramáticas en comparación con la transición de 2025 a 2030.

\clearpage

\begin{figure}[H]
    \centering
    \includegraphics[width=1\linewidth]{Figures/Resultados/comparativa_barras_individual_interval_meanHaltsPerVehicleWithin_estadio.pdf}
    \caption[Simulación de Paradas por Vehículo bajo Escenarios de Crecimiento Vehicular]{Simulación de Paradas por Vehículo bajo Escenarios de Crecimiento Vehicular. Fuente: Elaboración propia.}
    \label{fig:figure-paradas-vehiculo}
\end{figure}

El gráfico de barras presenta la comparativa de la tasa de detenciones (o paradas) por vehículo en la intersección crítica bajo los tres escenarios de demanda vehicular, 
sirviendo como métrica de la fricción operativa y la calidad del servicio del flujo. En el escenario actual de "Datos año 2025", cada vehículo experimenta un promedio de 
0.73 paradas en la intersección. La proyección para el "Datos año 2030" muestra un incremento marginal a 0.74 paradas por vehículo, indicando que la estructura semafórica o 
de control de la intersección mantiene una robustez notable frente al crecimiento vehicular anticipado. De manera crucial, el escenario extremo de "Datos año 20XX", con el 
parque vehicular duplicado, mantiene esta misma tasa de 0.74 paradas por vehículo, lo que sugiere que la saturación del flujo está primariamente afectando la duración de las 
detenciones (tiempo perdido) y la velocidad de desplazamiento, más que la frecuencia de las detenciones en sí mismas; en esencia, aunque los vehículos tienen que esperar más 
tiempo, la probabilidad de ser detenidos por unidad sigue siendo constante en los escenarios futuros analizados.

\clearpage

Una vez confirmado el punto como nodo crítico de la red vial a través del análisis de centralidad, se procedió a formular una estrategia de mitigación de la congestión. Dado 
que la intersección ya operaba bajo \textbf{control semafórico}, la intervención propuesta consistió en la reconfiguración geométrica de la misma, modelando específicamente la 
sustitución del semáforo por una glorieta (\textbf{rotonda}). Los resultados de la simulación de este nuevo escenario operacional se detallan a continuación.

\begin{figure}[H]
    \centering
    \includegraphics[width=1\linewidth]{Figures/Resultados/meanSpeedWithin_Año_2025_Semaforo.pdf}
    \caption[Impacto de la Rotonda en la Velocidad Media]{ Impacto de la Rotonda en la Velocidad Media. Fuente: Elaboración propia.}
    \label{fig:figure-impacto-rotonda-velocidad}
\end{figure}

El gráfico de barras presenta una comparativa de la velocidad media ($\text{m/s}$) de circulación en la intersección crítica de Comayagua bajo el volumen vehicular proyectado 
para el año 2025, contrastando el sistema de control semafórico existente con la propuesta de reconfiguración geométrica a rotonda. El escenario actual con semáforo registra 
una velocidad media de $\mathbf{3.21\ \text{m/s}}$, estableciendo la línea base de la eficiencia operativa del nodo. Por el contrario, la simulación de la implementación de la 
rotonda demuestra una mejora significativa en el flujo vehicular, elevando la velocidad media a $\mathbf{5.78\ \text{m/s}}$. Este incremento del $\approx 80\%$ en la velocidad 
de circulación valida la hipótesis de que la sustitución del control intermitente por el flujo continuo de la glorieta mitiga de manera sustancial la fricción del tráfico, 
mejorando la fluidez y el rendimiento del nodo crítico bajo las condiciones de demanda actuales proyectadas para 2025.

\clearpage

\begin{figure}[H]
    \centering
    \includegraphics[width=1\linewidth]{Figures/Resultados/meanTimeLossWithin_Año_2025_Semaforo.pdf}
    \caption[Impacto de la Rotonda en la Pérdida de Tiempo]{ Impacto de la Rotonda en la Pérdida de Tiempo. Fuente: Elaboración propia.}
    \label{fig:figure-impacto-rotonda-tiempo}
\end{figure}

El gráfico de barras compara la pérdida de tiempo promedio (en segundos) experimentada por los vehículos en la intersección crítica bajo el escenario de demanda vehicular 
proyectado para el año 2025, contrastando el control semafórico frente a la implementación de una glorieta. El escenario actual con semáforo registra una pérdida de tiempo 
de $\mathbf{23.53\ \text{s}}$, lo cual representa la ineficiencia generada por las detenciones obligatorias del ciclo semafórico. Por su parte, la simulación de la 
implementación de la rotonda muestra una reducción significativa de la congestión y de la fricción vehicular, disminuyendo el tiempo perdido a $\mathbf{17.87\ \text{s}}$. Esta 
diferencia de $5.66\ \text{s}$, que equivale a una reducción del $\approx 24\%$ en el tiempo de espera y detención, valida la rotonda como una medida altamente efectiva para 
mejorar el rendimiento operacional y la calidad de servicio en el nodo crítico.

\clearpage

\begin{figure}[H]
    \centering
    \includegraphics[width=1\linewidth]{Figures/Resultados/meanHaltsPerVehicleWithin_Año_2025_Semaforo.pdf}
    \caption[Impacto de la Rotonda en la Paradas por Vehículo]{ Impacto de la Rotonda en la Paradas por Vehículo. Fuente: Elaboración propia.}
    \label{fig:figure-impacto-rotonda-paradas}
\end{figure}

El gráfico de barras finaliza la comparativa de mitigación mostrando la tasa promedio de paradas por vehículo(\#) en la intersección crítica bajo el escenario de demanda del 
año 2025, contrastando el control semafórico frente a la implementación de la rotonda. Contrario a las métricas de velocidad y tiempo perdido, el análisis revela que la 
rotonda genera una mayor frecuencia de detenciones que el semáforo. El escenario con semáforo registra $\mathbf{0.73}$ paradas por vehículo, mientras que la simulación 
de la rotonda proyecta un aumento significativo a $\mathbf{1.49}$ paradas por vehículo. Este incremento en la tasa de detenciones es un hallazgo crítico, ya que si bien la 
rotonda mejora la velocidad media y reduce el tiempo perdido al mantener el flujo, su diseño obliga a una mayor cantidad de vehículos a detenerse o reducir la marcha 
significativamente para ceder el paso dentro del anillo, duplicando el número de paradas por unidad y afectando negativamente el confort de conducción y la eficiencia de 
combustibles.

Adicionalmente al nodo ya intervenido, y basándose en una inspección visual de campo y criterios de congestión preliminares, se identificaron otras tres intersecciones 
clave dentro de la red vial de Comayagua. Con el objetivo de evaluar su vulnerabilidad ante el crecimiento de la demanda, se aplicará el mismo protocolo de modelización 
comparativa que incluye el análisis del escenario base (actualidad), la proyección a 2030 y el escenario de máxima tensión (duplicación del parque vehicular, 20XX).

\begin{figure}[H]
    \centering
    \includegraphics[width=1\linewidth]{Figures/Resultados/comparativa_barras_combinada_my.pdf}
    \caption[Datos Comparativos de la intersección Bulevar 4 Centenario y Carretera RN-57]{ Datos Comparativos de la intersección Bulevar 4 Centenario y Carretera RN-57. Fuente: Elaboración propia.}
    \label{fig:figure-datos-comparativos-bulevar4C-carreteraRN57}
\end{figure}

Este conjunto de gráficos detalla la simulación del desempeño operacional de la intersección Bulevar 4 Centenario y Carretera RN-57, mediante las métricas de Velocidad Media, 
Tiempo Perdido y Paradas por Vehículo, a lo largo de los escenarios 2025, 2030 y 20XX. El análisis inicial revela un deterioro progresivo, pero moderado, en el flujo: la 
Velocidad Media cae de $\mathbf{1.77\ \text{m/s}}$ en 2025 a $\mathbf{1.71\ \text{m/s}}$ en 2030, mientras que el Tiempo Perdido se incrementa de 
$\mathbf{42.77\ \text{s}}$ a $\mathbf{43.68\ \text{s}}$. Sin embargo, la incongruencia crítica se manifiesta en el escenario de máxima demanda ($20\text{XX}$), donde 
la Velocidad Media ($1.76\ \text{m/s}$) y el Tiempo Perdido ($42.86\ \text{s}$) regresan a valores cercanos a la base de 2025; esta aparente "recuperación" es, 
en realidad, un diagnóstico de colapso estructural del sistema, interpretado por la simulación como la inmovilidad total del flujo o la desviación masiva de vehículos que 
abandonan la intersección, lo que evita que un volumen vehicular mayor sea procesado y estabiliza artificialmente las métricas internas de la vía, confirmando la extrema 
vulnerabilidad del nodo ante el crecimiento exponencial del parque vehicular.

\clearpage

\begin{figure}[H]
    \centering
    \includegraphics[width=1\linewidth]{Figures/Resultados/comparativa_barras_combinada_4CC0.pdf}
    \caption[Datos Comparativos de la intersección Bulevar 4 Centenario y Calle 0]{ Datos Comparativos de la intersección Bulevar 4 Centenario y Calle 0. Fuente: Elaboración propia.}
    \label{fig:figure-datos-comparativos-bulevar4C-calle0}
\end{figure}


Este conjunto de gráficos resume el análisis del desempeño operacional de la intersección Bulevar 4 Centenario y Carretera RN-57 a través de tres métricas de servicio bajo 
escenarios de demanda creciente (2025, 2030 y 20XX). El tránsito de 2025 a 2030 muestra el deterioro esperado, con la Velocidad Media cayendo de 
$\mathbf{2.71\ \text{m/s}}$ a $\mathbf{2.64\ \text{m/s}}$ y el Tiempo Perdido aumentando de $\mathbf{33.45\ \text{s}}$ a $\mathbf{37.14\ \text{s}}$; sin embargo, el 
escenario de máxima demanda ($20\text{XX}$) presenta una incongruencia crítica al mostrar una "mejora" artificial en las tres métricas, con la Velocidad Media aumentando 
a $\mathbf{3.08\ \text{m/s}}$ y el Tiempo Perdido disminuyendo a $\mathbf{28.63\ \text{s}}$. Esta inversión de tendencia en el escenario de mayor volumen vehicular se 
interpreta académicamente como la manifestación de un colapso del sistema simulado, donde la incapacidad de la intersección para procesar el tráfico extremo fuerza a los 
vehículos a la inmovilidad o a buscar rutas alternativas, estabilizando artificialmente las métricas internas e indicando un punto de vulnerabilidad estructural extrema.

\clearpage

\begin{figure}[H]
    \centering
    \includegraphics[width=1\linewidth]{Figures/Resultados/comparativa_barras_combinada_hst.pdf}
    \caption[Datos Comparativos de la intersección frente al Hospital Santa Teresa]{ Datos Comparativos de la intersección frente al Hospital Santa Teresa. Fuente: Elaboración propia.}
    \label{fig:figure-datos-comparativos-hospital-sta-teresa}
\end{figure}

El conjunto de gráficos detalla la simulación del desempeño operacional de la intersección Bulevar 4 Centenario y Carretera RN-57 mediante las métricas de Velocidad Media, 
Tiempo Perdido y Paradas por Vehículo, a lo largo de escenarios de demanda creciente (2025, 2030 y 20XX). La tendencia de 2025 a 2030 confirma el deterioro esperado por el 
incremento vehicular, con la Velocidad Media cayendo de $\mathbf{2.71\ \text{m/s}}$ a $\mathbf{2.64\ \text{m/s}}$ y el Tiempo Perdido ascendiendo de $\mathbf{33.45\ \text{s}}$ 
a $\mathbf{37.14\ \text{s}}$. Sin embargo, el escenario de máxima tensión ($20\text{XX}$) presenta una incongruencia crítica, pues el modelo arroja una "mejora" artificial, con 
la Velocidad Media aumentando a $\mathbf{3.08\ \text{m/s}}$ y el Tiempo Perdido cayendo a $\mathbf{28.63\ \text{s}}$; esta inversión de la tendencia no denota eficiencia, sino 
que diagnostica un colapso estructural simulado del nodo, donde el volumen de tráfico excede la capacidad de la vía y fuerza a los vehículos a la inmovilidad total o a la 
desviación masiva a rutas secundarias, lo que estabiliza artificialmente las métricas internas e indica un punto de vulnerabilidad extrema de la intersección ante la demanda 
futura.

\section{Interpretación y análisis de los resultados obtenidos}
La interpretación de los resultados de la simulación de tráfico se centra en dos ejes principales: la vulnerabilidad de la red ante el crecimiento vehicular proyectado y la eficacia 
de la intervención de mitigación propuesta.

\subsection{Análisis de Escenarios de Demanda Creciente}

El análisis de sensibilidad al crecimiento vehicular se realizó sobre una Intersección Antigua CA-5, la cual demostró ser un cuello de botella 
estructural identificado mediante la centralidad de intermediación. \\

\textbf{Intersección Antigua CA-5/RV217 bajo Demanda Futura:} \\
Los resultados (tiempo perdido, velocidad media y paradas por vehículo) de la Intersección A en los escenarios presentes (2025), proyectados (2030) y de máxima tensión (20XX \- duplicación)
muestran un deterioro gradual; no obstante, se presenta una anomalía significativa en el escenario extremo: \\

\begin{itemize}
    \item \textbf{Deterioro Esperado (2025-2030):} La proyección a 2030 indica un incremento moderado de la congestión. La Velocidad Media experimenta una ligera caída de 
    $\mathbf{3.21\ \text{m/s}}$ a $\mathbf{3.13\ \text{m/s}}$, mientras que la Pérdida de Tiempo aumenta de $\mathbf{23.53\ \text{s}}$ a $\mathbf{25.12\ \text{s}}$. La 
    tasa de Paradas por Vehículo se mantiene casi constante ($\mathbf{0.73}$ a $\mathbf{0.74}$), sugiriendo que la infraestructura actual absorbería el crecimiento 
    proyectado a corto plazo con un detrimento en la calidad de servicio.
    \item \textbf{Manifestación de Colapso (Escenario 20XX):} En el escenario de duplicación vehicular (20XX), el sistema alcanza su punto de saturación. Aunque la Pérdida de Tiempo 
    continúa ascendiendo a $\mathbf{25.45\ \text{s}}$, la Velocidad Media solo cae marginalmente a $\mathbf{3.00\ \text{m/s}}$. Esta baja sensibilidad al doble de volumen de 
    vehículos es una clara indicación del colapso operacional simulado. En este punto, el modelo predice que el flujo se estanca o que un volumen significativo de vehículos 
    opta por desviarse, evitando que las métricas reflejen un mayor deterioro interno. Este fenómeno subraya la extrema vulnerabilidad del nodo a niveles de demanda no 
    planificados.
\end{itemize}

\textbf{Intersección Bulevar 4 Centenario/RN-57 bajo Demanda Futura:} \\
Al aplicar el mismo análisis tri-métrico a la Intersección Bulevar 4 Centenario/RN-57, los resultados son más dramáticos y confirman la tendencia al colapso: \\
\begin{itemize}
    \item Tendencia de Congestión (2025-2030): La Velocidad Media se reduce de $\mathbf{2.71\ \text{m/s}}$ a $\mathbf{2.64\ \text{m/s}}$ (o de $\mathbf{1.64\ \text{m/s}}$ a 
    $\mathbf{1.94\ \text{m/s}}$ en otro gráfico, indicando variabilidad dependiendo del periodo o modelo), y el Tiempo Perdido aumenta significativamente de 
    $\mathbf{33.45\ \text{s}}$ a $\mathbf{37.14\ \text{s}}$. Este aumento del tiempo perdido es un indicativo directo de la ineficiencia generada por la gestión actual de la 
    intersección.
    \item Anomalía y Colapso (Escenario 20XX): En el escenario 20XX, las simulaciones muestran una inversión de tendencia, donde el Tiempo Perdido cae a 
    $\mathbf{28.63\ \text{s}}$ y la Velocidad Media se incrementa a $\mathbf{3.08\ \text{m/s}}$. Esta incongruencia es la evidencia más robusta del fenómeno de colapso: la 
    sobrecarga vehicular es tan alta que paraliza las arterias de acceso, obligando a una parte de los vehículos a desaparecer de la simulación (por inmovilidad o desviación 
    masiva), lo que reduce el volumen total de vehículos en movimiento dentro de la zona crítica y, artificialmente, "mejora" las métricas registradas. Este resultado es una 
    señal de alarma que pronostica fallas sistémicas a futuro.
\end{itemize}

\subsection{Evaluación de la Propuesta de Mitigación (Rotonda vs. Semáforo)}
La propuesta de intervención en la Intersección Antigua CA-5/RV217 consistió en la sustitución del control semafórico por una glorieta, evaluada con datos del escenario actual 
(2025) para aislar el impacto de la mejora geométrica.

\begin{itemize}
    \item \textbf{Impacto en la Fluidez (Velocidad Media y Tiempo Perdido):} 
    \begin{itemize}
    \item La Velocidad Media se incrementó drásticamente de $\mathbf{3.21\ \text{m/s}}$ (semáforo) a $\mathbf{5.78\ \text{m/s}}$ (rotonda). Esta mejora del 
    $\approx 80\%$ confirma que el flujo continuo de la rotonda elimina las detenciones totales inherentes al ciclo semafórico, promoviendo una circulación más rápida.
    \item La Pérdida de Tiempo se redujo de $\mathbf{23.53\ \text{s}}$ (semáforo) a $\mathbf{17.87\ \text{s}}$ (rotonda). Esta reducción del $\approx 24\%$ valida la rotonda 
    como una solución altamente efectiva para mitigar la congestión temporal.
\end{itemize}
    \item \textbf{Impacto en la Fricción (Paradas por Vehículo):}
    \begin{itemize}
        \item El resultado más crucial fue el aumento en la tasa de Paradas por Vehículo, la cual se duplicó de $\mathbf{0.73}$ (semáforo) a $\mathbf{1.49}$ (rotonda). Esta 
        métrica sugiere que, si bien la rotonda promueve la velocidad y reduce el tiempo total de espera, lo hace a expensas de forzar a un mayor número de vehículos a 
        realizar detenciones o reducciones de marcha significativas para ceder el paso dentro del anillo. Esto impacta negativamente el confort de conducción y potencialmente 
        la eficiencia de combustible por la mayor cantidad de ciclos de aceleración-desaceleración.
    \end{itemize}
\end{itemize}

\section{Comparación con Investigaciones Previas o Estándares}
Para contextualizar los resultados obtenidos en Comayagua, la evaluación del desempeño de las intersecciones se contrasta con los \acrfull{nds} definidos por el 
\Acrshort{hcm} y la literatura de ingeniería de transporte, utilizando principalmente la Velocidad Media y la Pérdida de Tiempo como métricas de referencia.

\subsection{Niveles de Servicio (NdS) y Vulnerabilidad}
Los niveles de servicio (NdS) se dividen desde el nivel A (flujo libre) hasta el F (congestión intensa o colapso).  Las velocidades medias logradas en las intersecciones 
críticas, sobre todo en los contextos futuros, son un signo de un servicio deficiente:

\begin{table}[h!]
\centering
\resizebox{\textwidth}{!}{
\begin{tabular}{lcccc}
\hline
\textbf{Intersección/Métrica} &
\textbf{Escenario Base (2025)} &
\textbf{Escenario 2030} &
\textbf{Escenario 20XX} &
\textbf{NdS Estimado} \\
\hline

\textbf{Intersección A (Velocidad)} &
\textbf{3.21 m/s} &
\textbf{3.13 m/s} &
\textbf{3.00 m/s} &
D / E \\

\textbf{Bv. 4 Centenario/RN-57 (Tiempo Perdido)} &
\textbf{33.45 s} &
\textbf{37.14 s} &
\textbf{28.63 s (Colapso)} &
E / F \\
\hline
\end{tabular}}
\caption{Comparación del desempeño por escenarios y nivel de servicio estimado.}
\label{tab:comparacion-escenarios}
\end{table}

Los valores de Pérdida de Tiempo en el Bulevar 4 Centenario/RN-57 sobrepasan los $35\ \text{s}$ en la proyección del año 2030, lo que pone a este cruce al borde del nivel de 
servicio E o F. De acuerdo con el \Acrshort{hcm}, esto se trata de un umbral de funcionamiento inaceptable, caracterizado por retrasos extensos y una elevada inestabilidad 
del flujo.  La tendencia hacia el colapso observada en el escenario 20XX de las dos intersecciones concuerda con la transición de \acrshort{nds} E a F, en la cual el sistema 
pierde la capacidad de procesar la demanda y se produce una desviación del tráfico o una inmovilidad total.

\subsection{Comparación de Soluciones: Semáforo vs. Rotonda}
La sugerencia de reemplazar el control semafórico por una rotonda en la intersección A contrasta con las tendencias documentadas a nivel global, que prefieren la rotonda bajo ciertas 
circunstancias:

\begin{itemize}
    \item Velocidad Media y Flujo: La rotonda elevó la Velocidad Media a $\mathbf{5.78\ \text{m/s}}$, un valor que se acerca al NdS B (flujo estable con detenciones mínimas) y 
    supera significativamente el NdS D/E del semáforo. Esto concuerda con investigaciones que demuestran que las glorietas, al reemplazar las detenciones programadas por 
    conflictos de cesión de paso, incrementan la velocidad promedio del tramo.
    \item Pérdida de Tiempo: La reducción del $\approx 24\%$ en el tiempo perdido es coherente con estudios de caso en Europa y Norteamérica, donde la eliminación de los tiempos 
    muertos del ciclo semafórico (luz roja) resulta en una disminución global de la demora.
    \item Paradas por Vehículo (Fricción): El aumento de las paradas de $\mathbf{0.73}$ a $\mathbf{1.49}$ contradice parcialmente el objetivo de reducción de fricción, pero es un resultado 
    típico en rotondas de alto volumen. Mientras que el semáforo detiene a todos los vehículos de forma intermitente, la rotonda detiene solo a los vehículos en las entradas 
    (para ceder el paso), resultando en que una mayor proporción de vehículos tienen que detenerse en algún punto (frecuencia de paradas alta), aunque la duración total de la 
    detención sea mucho menor (tiempo perdido bajo). En términos de seguridad y confort, este aumento en las maniobras y detenciones es un factor que se debe evaluar frente a 
    los beneficios de fluidez.
\end{itemize}

En conclusión, la comparación confirma que el rendimiento actual y futuro de las intersecciones críticas en Comayagua se sitúa en niveles de servicio que requieren intervención 
inmediata (NdS D a F), y que la solución de rotonda, si bien introduce fricción operacional (más paradas), es una solución de alto impacto para restaurar el NdS a rangos 
funcionales (B/C).

\section{Validación de objetivos planteados}
Los objetivos del proyecto se cumplieron satisfactoriamente mediante la siguiente secuencia de resultados:

\begin{itemize}
    \item \textbf{Diagnóstico y Puntos Críticos (Obj. 1, 2, 3):} Se modeló la red como un grafo, utilizando la Centralidad de Intermediación para identificar los nodos de alta 
    dependencia. Los resultados confirmaron su nivel de congestión (NdS E/F).
    \item \textbf{Modelización de Escenarios (Obj. 4):} Se utilizaron algoritmos de la teoría de grafos y herramientas de simulación de tráfico para reproducir y evaluar los escenarios 
    2030 y 20XX. El escenario extremo (20XX) en ambas intersecciones reveló un colapso operacional simulado, validando que el NdS será inaceptable en el futuro.
    \item \textbf{Propuesta de Estrategias (Obj. 5):} Se evaluó la reconfiguración semafórica por una rotonda en la Intersección Antigua CA-5/RV217, confirmando la viabilidad de una solución de flujo continuo para mejorar la movilidad.
\end{itemize}

