\chapter{Planificación Académica y Distribución de Cargas}
\label{cp:Planificacion-Cargas}


\insertminitoc
\parindent0pt

\section{El proceso de programación de la oferta académica}
El proceso de programación de la oferta académica se transformó de una tarea administrativa rutinaria a un procedimiento estratégico de alta complejidad. Históricamente, la planificación se ejecutó mediante métodos manuales basados en la repetición de patrones históricos; sin embargo, la masificación de la matrícula y la diversificación curricular exigieron la adopción de modelos dinámicos. Se estableció que la programación eficiente no solo implicó asignar espacios y tiempos, sino sincronizar múltiples variables críticas disponibilidad docente, infraestructura física y demanda estudiantil en un sistema coherente que garantizara la operatividad institucional y la calidad del servicio educativo.

En el contexto de la modernización institucional, Yang \cite{yang_exploration_2024} exploró la ruta de transformación digital en la gestión educativa universitaria. Su investigación cuantificó el impacto de digitalizar los procesos administrativos, determinando que la implementación de sistemas inteligentes redujo el tiempo dedicado a la gestión en un 30\%. Además, se observó una correlación directa entre la agilidad administrativa y la percepción de calidad; el estudio reportó un incremento del 25\% en la satisfacción estudiantil y una mejora del 20\% en las evaluaciones de calidad educativa. Estos hallazgos validaron que la optimización del proceso de programación no fue meramente una mejora técnica, sino un factor determinante para la eficiencia operativa global de la universidad.

A pesar de las ventajas de la digitalización, se identificó que la persistencia de métodos manuales constituyó la principal barrera para la eficiencia. Farinola y Assogba \cite{farinola_explicit_2025} analizaron la generación de horarios en instituciones de educación superior, señalando que la preparación manual resultó ser un proceso propenso a errores y extremadamente consumidor de tiempo. Se evidenció que los planificadores humanos, al enfrentarse a restricciones conflictivas (como la disponibilidad de aulas versus la preferencia docente), tendieron a producir soluciones subóptimas que requirieron ajustes continuos durante el periodo académico. La investigación concluyó que la automatización mediante inteligencia artificial fue necesaria para eliminar la redundancia y garantizar cronogramas libres de conflictos desde la primera iteración.

Para superar la subjetividad en la planificación, se recurrió a técnicas avanzadas de tratamiento de datos. Almaghrabi et al. \cite{almaghrabi_sok_2024} presentaron una sistematización del conocimiento (Systematization of Knowledge - SoK) sobre el impacto de la Minería de Datos Educativos (EDM) en la administración organizacional. Se demostró que la aplicación de algoritmos de minería permitió transitar de una gestión basada en la intuición a una basada en evidencia. El estudio destacó que las técnicas de agrupamiento (\textit{clustering}) y predicción facilitaron a los administradores la identificación de patrones ocultos en el comportamiento de matrícula, permitiendo ajustar la oferta académica a las necesidades reales de los estudiantes y optimizar la asignación de recursos antes del inicio del ciclo lectivo.

La operatividad del proceso de programación dependió críticamente de la interconexión entre plataformas heterogéneas. Pérez-Jorge et al. \cite{perez-jorge_impact_2025} evaluaron el impacto de las Interfaces de Programación de Aplicaciones (APIs) impulsadas por IA en la gestión de información educativa. Se estableció que la fragmentación de datos entre los Sistemas de Gestión del Aprendizaje (LMS) y los sistemas administrativos (ERP) generó inconsistencias en la oferta académica. La investigación determinó que el uso de APIs inteligentes permitió un flujo de datos en tiempo real, facilitando que la programación de la oferta se alimentara automáticamente de los registros de rendimiento y prerrequisitos estudiantiles, eliminando la necesidad de la captura manual de datos y reduciendo la latencia en la toma de decisiones.

La convergencia de la automatización y la integración de datos generó un impacto multidimensional en la institución. Al contrastar los hallazgos de Farinola y Assogba \cite{farinola_explicit_2025} con los de Yang \cite{yang_exploration_2024}, se observó que la implementación de generadores de horarios basados en IA no solo resolvió el problema logístico, sino que mejoró el clima organizacional. Mientras Farinola y Assogba \cite{farinola_explicit_2025} demostraron que la automatización eliminó los conflictos de horarios (choques de aulas o docentes), Yang \cite{yang_exploration_2024} confirmó que esta eficiencia técnica se tradujo directamente en una mayor satisfacción de los involucrados. Ambos estudios coincidieron en que la tecnología liberó al personal administrativo de tareas repetitivas, permitiéndoles enfocarse en la gestión estratégica y la atención al estudiante.

Sin embargo, se reconoció que la automatización del proceso enfrentó un desafío matemático inherente. Retomando a Abdipoor et al. \cite{abdipoor_meta-heuristic_2023}, se reiteró que el Problema de Horarios Universitarios (UCTP) pertenece a la clase de complejidad NP-Hard. En el contexto del proceso de programación, esto significó que buscar la ``oferta académica perfecta'' implicó explorar un espacio de soluciones que crecía exponencialmente con cada nueva asignatura o sección agregada. Se concluyó que los métodos de programación lineal tradicionales fueron insuficientes para manejar esta explosión combinatoria en tiempos razonables, validando la necesidad de incorporar metaheurísticas en el núcleo del motor de programación para obtener soluciones factibles en tiempos operativos viables.

La evolución del proceso de programación permitió incorporar la sostenibilidad financiera como una variable de control. Almaghrabi et al. \cite{almaghrabi_sok_2024} destacaron en su revisión que la minería de datos educativos facultó a las instituciones para predecir la asignación de recursos con alta precisión. Se observó que, al analizar los patrones históricos de inscripción y deserción, los algoritmos pudieron sugerir el número óptimo de secciones a abrir, evitando tanto la saturación de aulas como la subutilización de espacios. Esta capacidad predictiva transformó la programación de la oferta de un ejercicio reactivo a uno proactivo, alineando la disponibilidad académica con las restricciones presupuestarias y de infraestructura de la universidad.

Asimismo, el proceso de programación evolucionó hacia un enfoque centrado en el usuario final. Pérez-Jorge et al. \cite{perez-jorge_impact_2025} argumentaron que la gestión de información impulsada por IA permitió personalizar la experiencia educativa. Se determinó que, mediante el análisis de las trayectorias académicas individuales accesibles a través de APIs, el sistema de programación pudo priorizar la apertura de asignaturas críticas para el egreso de cohortes específicas. Esto aseguró que la oferta académica no fuera estática, sino que se adaptara dinámicamente a las necesidades de avance curricular de los estudiantes, reduciendo los tiempos de graduación y mejorando los indicadores de eficiencia terminal.

En conclusión, la reingeniería del proceso de programación de la oferta académica se fundamentó en la integración de tres pilares tecnológicos: la minería de datos para la predicción de la demanda, las APIs para la interoperabilidad de sistemas y los algoritmos metaheurísticos para la resolución de conflictos. La evidencia analizada confirmó que la transición hacia este modelo automatizado fue indispensable para gestionar la complejidad de la educación superior moderna. Se estableció que solo mediante esta simbiosis tecnológica fue posible garantizar una oferta académica que fuera simultáneamente eficiente en costos, viable logísticamente y pertinente para las necesidades estudiantiles.

\section{La Cadencia Académica y el flujo curricular}
La cadencia académica se definió no solo como la cronología de los periodos lectivos, sino como la velocidad y fluidez con la que el estudiantado transitó a través de la malla curricular. Se identificó que el flujo curricular no fue un proceso lineal uniforme, sino un sistema dinámico propenso a interrupciones causadas por la falta de sincronización entre la oferta institucional y el avance real de los estudiantes. El análisis de esta variable fue determinante, pues se estableció que cualquier desajuste en la programación de asignaturas críticas generó un ``efecto dominó'', alterando los tiempos de egreso y provocando la acumulación de matrícula en niveles inferiores, lo que saturó la capacidad operativa de la universidad.

Para comprender la estructura subyacente de este flujo, se retomó el enfoque de Stavrinides y Zuev \cite{stavrinides_course-prerequisite_2023}, quienes modelaron los planes de estudio mediante Redes de Prerrequisitos de Cursos (CPNs). Se analizó la malla curricular no como un listado, sino como un grafo dirigido acíclico donde los nodos representaron las asignaturas y los enlaces sus dependencias . La investigación determinó que la complejidad del flujo curricular residió en la ``centralidad'' de ciertos nodos; se demostró que ignorar la topología de la red al programar la oferta académica provocó bloqueos estructurales, impidiendo que los estudiantes avanzaran a pesar de tener cupos disponibles en materias no correlativas.

La gestión eficiente de este flujo requirió una infraestructura de datos robusta capaz de almacenar la trazabilidad histórica del estudiante. Amo et al. \cite{amo_educational_2021} propusieron una arquitectura de Almacén de Datos Educativos (EDW) modular y basada en la nube para centralizar esta información dispersa. Se implementó un diseño que permitió la integración de datos provenientes de Sistemas de Gestión del Aprendizaje (LMS) y sistemas administrativos, facilitando el análisis de grandes volúmenes de datos (Big Data). El estudio concluyó que contar con un repositorio unificado y seguro fue el prerrequisito técnico indispensable para aplicar algoritmos de análisis sobre el progreso curricular, garantizando que la toma de decisiones se basara en la realidad histórica de los expedientes y no en proyecciones teóricas.

Complementariamente, se abordó la predicción del comportamiento estudiantil dentro de este flujo. Almalawi et al. \cite{almalawi_predictive_2024} realizaron una revisión sistemática de modelos predictivos con fines educativos, identificando que el aprendizaje automático supervisado fue la técnica más eficaz para anticipar el rendimiento y la deserción. Se observó que los modelos predictivos permitieron identificar patrones de riesgo en etapas tempranas del flujo curricular, alertando a los gestores sobre qué estudiantes tenían alta probabilidad de reprobar o abandonar. Esta capacidad de anticipación fue clave para la planificación académica, pues permitió ajustar la oferta de cursos remediales o tutorías antes de que el flujo se interrumpiera definitivamente.

\subsection{Importancia de la secuencia de asignaturas}
La secuencia de asignaturas se identificó como la columna vertebral de la planificación académica, regida por una jerarquía estricta. Stavrinides y Zuev \cite{stavrinides_course-prerequisite_2023} introdujeron el concepto de ``estratificación topológica'' para clasificar las asignaturas según su nivel de profundidad y dependencia. Se determinó que violar esta secuencia natural —por ejemplo, ofertar cupos en materias avanzadas sin garantizar la cobertura suficiente en sus prerrequisitos— generó inconsistencias académicas. El análisis de la red curricular reveló que ciertas asignaturas actuaron como ``cuellos de botella'' estructurales; su reprobación o falta de oferta detuvo el avance de cohortes enteras, validando la necesidad de priorizar estos cursos en la asignación de recursos.

La secuencia óptima no dependió únicamente de los prerrequisitos formales, sino también de la capacidad del estudiante para gestionar su carga cognitiva. Bhosale y Hore \cite{prof_ramkrishna_more_college_pradhikaran_pune_india_ai-based_2025} desarrollaron un motor de asignación de tiempo basado en IA que consideró la ``asignación inteligente del tiempo de estudio''. Aunque su enfoque se centró en la planificación personal del alumno, su modelo demostró que el éxito académico depende de equilibrar la carga de trabajo en función de la disponibilidad real. Esto sugiere que la oferta académica no debe estructurarse solo por bloques administrativos rígidos, sino considerar heurísticas que eviten la combinación inviable de múltiples asignaturas de alta complejidad en un mismo ciclo temporal.

La interrelación entre la estructura secuencial del currículo y el riesgo de fracaso escolar fue validada al cruzar los hallazgos de Stavrinides y Zuev \cite{stavrinides_course-prerequisite_2023} con los de Almalawi et al. \cite{almalawi_predictive_2024}. Mientras que el análisis topológico de Stavrinides y Zuev \cite{stavrinides_course-prerequisite_2023} permitió identificar qué asignaturas eran críticas para mantener la conectividad del grafo curricular, los modelos predictivos revisados por Almalawi et al. \cite{almalawi_predictive_2024} cuantificaron la probabilidad de que un estudiante fallara en dichos nodos críticos basándose en su historial. Ambos estudios coincidieron en que la gestión de la oferta académica debió focalizarse en los puntos de intersección entre la alta complejidad estructural y el alto riesgo predictivo, asegurando que la secuencia de asignaturas no se convirtiera en un filtro excluyente sino en un camino transitable.

\subsection{Tiempos y ciclos de gestión académica}
Los tiempos de dedicación académica dejaron de ser variables estáticas para adaptarse a las necesidades del estudiante moderno. Bhosale y Hore \cite{prof_ramkrishna_more_college_pradhikaran_pune_india_ai-based_2025} propusieron un enfoque de planificación centrado en la gestión del tiempo personal, donde el sistema generó horarios de estudio y recordatorios basándose en la disponibilidad real del usuario mediante almacenamiento local y reglas heurísticas. Este enfoque evidenció que alinear las exigencias académicas con la disponibilidad temporal del estudiante redujo la ansiedad y mejoró la organización personal, contrastando con los modelos tradicionales que asumen una disponibilidad ilimitada del alumnado.

La gestión de los datos durante los ciclos académicos implicó el procesamiento masivo de información sensible. Amo et al. \cite{amo_educational_2021} enfatizaron la importancia de la privacidad y la seguridad en la arquitectura del almacén de datos. Se diseñó un sistema que incorporó técnicas de anonimización y encriptación para proteger la identidad estudiantil mientras se analizaban sus patrones de actividad. El estudio determinó que garantizar la integridad y confidencialidad de los datos fue fundamental para mantener la confianza institucional y cumplir con las normativas éticas (como GDPR) al procesar información para la toma de decisiones.

Finalmente, la definición de la oferta requirió integrar la probabilidad de éxito en la ecuación de planificación. Retomando la revisión de Almalawi et al. \cite{almalawi_predictive_2024}, se destacó que los algoritmos de clasificación (como Random Forest y Naive Bayes) permitieron prever el rendimiento estudiantil con alta precisión. La integración de estas predicciones en la gestión académica facultó a la institución para anticipar la demanda de recursos de apoyo en periodos específicos, distribuyendo la carga docente y de tutorías de manera proactiva para mitigar las tasas de deserción detectadas por los modelos.

\section{Problemáticas en la distribución de secciones}
La distribución de secciones se identificó como el punto crítico donde la planificación estratégica convergió con la realidad operativa. Una vez definida la oferta académica macro, el desafío administrativo consistió en determinar la cantidad exacta de paralelos (grupos) necesarios para cada asignatura. Se observó que este proceso fue vulnerable a sesgos históricos; la repetición inercial de la programación de años anteriores, sin considerar las variaciones estocásticas de la demanda actual, derivó frecuentemente en desequilibrios estructurales. La ineficiencia en esta etapa no solo generó conflictos logísticos, sino que comprometió la calidad educativa al forzar la apertura de cursos sobrepoblados o, inversamente, diluir los recursos docentes en secciones con baja rentabilidad social.

En el intento de racionalizar esta distribución, Pauta Riera et al. \cite{riera_pronostico_2025} realizaron un estudio comparativo en la Universidad Católica de Cuenca sobre el uso de modelos de regresión para pronosticar la demanda estudiantil. Se aplicaron modelos lineales, logarítmicos y polinómicos para predecir la matrícula en carreras de ingeniería. La investigación determinó que, si bien los modelos estadísticos clásicos ofrecieron una aproximación inicial, los modelos lineales tendieron a simplificar excesivamente el comportamiento de la matrícula, presentando coeficientes de determinación ($R^2$) inferiores a los modelos polinómicos ($0.96$). Se concluyó que confiar exclusivamente en proyecciones lineales para la distribución de secciones condujo a errores de subestimación en periodos de crecimiento acelerado, resultando en una falta de cobertura para la demanda real.

La inexactitud en la predicción de la cantidad de estudiantes tuvo repercusiones directas en la eficiencia administrativa. Retomando a Shao et al. \cite{shao_machine_2021}, se estableció que la incapacidad para pronosticar las tasas de inscripción con precisión generó costos operativos innecesarios y una carga administrativa adicional. Se evidenció que cuando la distribución de secciones se basó en estimaciones incorrectas, la administración se vio obligada a realizar ajustes reactivos de ``último minuto'', como la contratación emergente de docentes o la reasignación forzada de aulas. El estudio enfatizó que minimizar el error de predicción fue fundamental para asignar los recursos apropiados —asientos, espacio de laboratorio y auxiliares— antes del inicio del ciclo, evitando la improvisación logística.

Finalmente, la problemática de distribución trascendió lo administrativo para afectar la permanencia estudiantil. Shilbayeh y Abonamah \cite{shilbayeh_predicting_2021} desarrollaron modelos predictivos para identificar patrones de deserción, demostrando que el abandono estudiantil no es un evento aleatorio, sino un fenómeno predecible vinculado a variables demográficas y financieras específicas. Su investigación reveló que ciertos perfiles estudiantiles presentan un riesgo de deserción inherente más alto, lo que implica que la distribución de secciones no debe ser uniforme. Por tanto, se determinó que la planificación de cupos debe priorizar estratégicamente a estos grupos vulnerables, asegurando que la disponibilidad de horarios no se convierta en una barrera adicional para estudiantes que ya enfrentan desafíos socioeconómicos.

\subsection{Subestimación de cupos y saturación}
La subestimación de la demanda real en la planificación de cupos derivó en el fenómeno de ``saturación áulica''. Pauta Riera et al. \cite{riera_pronostico_2025} evidenciaron que la falta de herramientas predictivas robustas provocó que la infraestructura física se viera rebasada por la matrícula efectiva. En su análisis de la Unidad Académica de Ingeniería, se detectó que la discrepancia entre las plazas ofertadas (basadas en históricos estáticos) y la demanda real generó sobrepoblación en las aulas. Esta saturación no solo contravino las normas pedagógicas de ratio estudiante/docente, sino que obligó a la institución a improvisar desdobles de grupos sin la planificación de infraestructura adecuada, comprometiendo la calidad del proceso de enseñanza-aprendizaje.

Para corregir esta subestimación, fue necesario identificar qué variables determinaban realmente la ocupación de un curso. Shao et al. \cite{shao_machine_2021} aplicaron métricas de ``importancia de variables'' (\textit{variable importance}) derivadas de algoritmos de Random Forest, descubriendo factores que los métodos tradicionales ignoraban. Se determinó que variables como la ``especialidad del estudiante'' (\textit{major}) y su ``nivel de clase'' (\textit{freshman}, \textit{sophomore}, etc.) tuvieron un peso predictivo superior al historial simple de inscripciones pasadas. El estudio concluyó que la subestimación de cupos ocurrió frecuentemente porque los planificadores humanos no lograron ponderar las interacciones no lineales entre estas variables demográficas y académicas, asumiendo erróneamente una demanda homogénea.

La superioridad de los enfoques avanzados para mitigar la saturación se confirmó al contrastar los resultados de Pauta Riera et al. \cite{riera_pronostico_2025} con los de Shao et al. \cite{shao_machine_2021}. Mientras que Pauta Riera et al. \cite{riera_pronostico_2025} lograron mejorar el ajuste de la oferta mediante modelos de regresión polinómica ($R^2 \approx 0.96$), Shao et al. \cite{shao_machine_2021} demostraron que los métodos de aprendizaje automático (Random Forest) redujeron aún más el error cuadrático medio (RMSE) al manejar datos complejos y ruidosos. Ambos estudios coincidieron en que la planificación manual o basada en promedios simples fue sistemáticamente deficiente para anticipar picos de demanda. Se estableció que la única vía para evitar la subestimación de cupos y la consecuente saturación fue la adopción de modelos matemáticos no lineales capaces de simular el comportamiento de inscripción con alta fidelidad.

\subsection{Impacto de la mala distribución en el egreso estudiantil}
La correlación entre la disponibilidad de cursos y la deserción universitaria se analizó sistemáticamente en la literatura reciente. Quimiz-Moreira et al. \cite{quimiz-moreira_factors_2025} realizaron una revisión sistemática (2012-2024) sobre los factores de deserción universitaria, clasificándolos en dimensiones individuales, académicas e institucionales. Se identificó que, dentro de los factores institucionales, la gestión ineficiente de la oferta académica jugó un rol preponderante. El estudio reveló que la incapacidad de la institución para proveer una ruta curricular fluida —debido a la mala distribución de secciones o horarios conflictivos— actuó como un detonante para el abandono, especialmente en estudiantes que ya presentaban vulnerabilidad académica. Se concluyó que la predicción de la deserción debió integrar variables de gestión institucional y no culpar únicamente al rendimiento del estudiante.

Más allá de los factores macro, la experiencia diaria del estudiante con los servicios universitarios resultó ser un predictor crítico. Matz et al. \cite{matz_using_2023} investigaron la retención estudiantil integrando datos de nivel ``meso'', que capturaron la interacción del estudiante con el entorno universitario a través de aplicaciones móviles y sistemas. Se descubrió que la falta de compromiso (\textit{engagement}) con los servicios institucionales —evidenciada por una baja frecuencia de interacción digital— fue una señal temprana de alerta. El modelo de Random Forest utilizado demostró que combinar datos demográficos con métricas de comportamiento (como el uso de la app universitaria) mejoró significativamente la predicción de la retención. Esto sugirió que una distribución de secciones que no fomente la interacción constante del estudiante con su entorno académico debilita su sentido de pertenencia y precipita su salida.

Finalmente, se estableció la necesidad de utilizar la distribución de secciones como una herramienta de intervención preventiva. Shilbayeh y Abonamah \cite{shilbayeh_predicting_2021} propusieron un enfoque donde la identificación de estudiantes en riesgo de deserción informara la planificación académica. Al utilizar algoritmos para clasificar a los estudiantes según su probabilidad de abandono, la institución pudo priorizar la asignación de cupos en secciones críticas para estos perfiles vulnerables. Se determinó que una distribución ``inteligente'' no solo buscó llenar aulas, sino garantizar que los estudiantes con mayor riesgo de \textit{attrition} tuvieran acceso garantizado a los cursos necesarios para mantener su progreso, transformando la gestión de horarios en un mecanismo activo de retención y éxito estudiantil.