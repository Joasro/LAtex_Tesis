\chapter[Contexto Específico del Estudio]{Contexto Específico del Estudio}
\label{cp:contexto-especifico}

\insertminitoc
\parindent0pt

\section{Panorama de la movilidad vehicular en Honduras}
\label{sec:panorama-movilidad-honduras}
En Honduras, la congestión vehicular es consecuencia de elementos sociales, estructurales y económicos 
que contribuyen a que se contamine más y disminuya la eficacia del transporte en las ciudades. El parque 
automotor ha crecido de manera sostenida durante la última década, sobre todo en ciudades relevantes como 
Tegucigalpa, Comayagua, y San Pedro Sula. En estas urbes, el crecimiento urbano ha rebasado la capacidad 
para planificar el transporte y las vías. En una entrevista de 2023 con \cite{benavidez_medidas_2023}, el alcalde de 
Tegucigalpa, Jorge Aldana, reveló datos críticos sobre la movilidad de la capital. Mencionó que la ciudad 
cuenta con una red vial pavimentada de 841 km, una infraestructura diseñada para soportar idealmente el 
tránsito de unos 250,000 vehículos. Sin embargo, a raíz de la pandemia de COVID-19, la adquisición de 
automóviles se disparó, resultando en un parque vehicular matriculado de aproximadamente 600,000 unidades 
en esa fecha, lo que representa un incremento del 7.8\% y una sobrecarga significativa para la infraestructura existente.

Según \acrfull{ine} \cite{noauthor_parque_nodate}, en Honduras al culminar el año 2021, las importaciones de vehículos aumento a 880 millones 
de dólares, con esto se registra un aumento del 94\% en comparación al año anterior, ya que en 2020 las 
importaciones rondaban los 454 millones de dólares. Al finalizar el año 2021, el parque vehicular aumento 
a 2,415,192 unidades registrando un incremento del 10\%.

\begin{figure}[!htbp]
    \centering
    \includegraphics[width=0.7\linewidth]{Figures/evo_parque_vehicular.pdf}
    \caption[Evolución del parque vehicular en Honduras.]{Evolución del parque vehicular en Honduras. Fuente: \acrshort{ine} en base a datos extraído del Instituto de la Propiedad}
    \label{fig:figure-evo-parque-vehicular-06}
\end{figure}

En la ciudad de Tegucigalpa, se aplicó un plan de alivio vial, ya que una buena parte del parque 
vehicular es ocupado por empleadores públicos, a los cuales se les impuso una medida inmediata la cual es 
“teletrabajo”. Teniendo como objetivo disminuir en un 25\% la carga vehicular diaria \cite{lopez_trafico_nodate}.

\section{Características del tráfico en la ciudad de Comayagua}
\label{sec:caracteristicas-trafico-comayagua}
Ubicada en el centro de Honduras, Comayagua se beneficia de una posición geográfica favorable al estar 
ubicada justo sobre la carretera CA-5, que es una de las principales rutas del país y enlaza a Tegucigalpa 
con San Pedro Sula.  Esta posición estratégica genera un tráfico mixto y significativo, que incluye el 
tránsito vehicular local, el transporte de pasajeros entre ciudades, el tránsito nacional y el tránsito 
pesado de carga.
Comayagua posee unas dinámicas peculiares en el flujo vial, en las horas de menor demanda, el tráfico 
local suele tener una densidad baja; no obstante, en los horarios de la mañana y la tarde, el incremento 
del flujo interurbano se combina con el tráfico local, causando congestión en las intersecciones clave. 
Consideramos que esta dualidad inherente “funcionar como ciudad tanto de destino como de paso” exige la 
necesidad de establecer modelos de gestión que puedan distinguir entre el flujo interno y el de tránsito. 
Este requisito introduce una complejidad considerable al diseño de sistemas de control de semáforos, a la 
estrategia de rutas alternativas y a la priorización del tráfico.
Comayagua experimentó un crecimiento sostenido en su parque vehicular entre 2017 y 2021, pasando de 
93,558 a 135,376 unidades durante ese periodo, lo que representa un incremento del 45\% y muestra una 
expansión significativa en la cantidad de vehículos registrados año tras año, a pesar de que su volumen 
total sigue siendo inferior al de departamentos como Francisco Morazán y Cortés, esta tendencia evidencia 
que el dinamismo económico y la urbanización están impulsando la movilidad y el desarrollo en la 
región \cite{noauthor_parque_nodate}.

\begin{longtable}[c]{>{\raggedright\arraybackslash}p{3.5cm} *{5}{>{\centering\arraybackslash}p{1.5cm}}}
\caption{Parque vehicular clasificado por departamentos}
\label{tab:table-03} \\
\toprule
\textbf{Departamento} & \textbf{2017} & \textbf{2018} & \textbf{2019} & \textbf{2020} & \textbf{2021} \\
\midrule
\endfirsthead

\multicolumn{6}{c}{{\textit{\bfseries Tabla \thetable\ (continuación)}}} \\
\toprule
\textbf{Departamento} & \textbf{2017} & \textbf{2018} & \textbf{2019} & \textbf{2020} & \textbf{2021} \\
\midrule
\endhead

\midrule
\multicolumn{6}{r}{{\textit{Continúa en la siguiente página}}} \\
\endfoot

\bottomrule
\endlastfoot

Francisco Morazán   & 488,373 & 531,934 & 568,329 & 597,857 & 627,288 \\
Cortés              & 150,000 & 152,000 & 155,000 & 158,000 & 160,000 \\
Atlántida           & 112,597 & 125,337 & 140,571 & 151,284 & 165,387 \\
Yoro                & 100,748 & 113,052 & 123,492 & 132,651 & 149,622 \\
Comayagua           & 93,558  & 103,111 & 111,665 & 119,397 & 135,376 \\
Olancho             & 83,788  & 94,687  & 104,952 & 113,543 & 128,872 \\
Choluteca           & 65,313  & 73,150  & 80,578  & 86,826  & 97,744  \\
El Paraíso          & 61,823  & 67,667  & 73,409  & 79,012  & 90,961  \\
Colón               & 52,944  & 60,329  & 68,750  & 76,624  & 90,762  \\
Copán               & 54,551  & 61,116  & 66,358  & 71,544  & 80,272  \\
Santa Bárbara       & 50,398  & 55,841  & 61,002  & 66,018  & 76,676  \\
Lempira             & 27,588  & 31,034  & 34,414  & 37,945  & 44,748  \\
Valle               & 24,990  & 28,274  & 30,924  & 32,948  & 37,509  \\
Ocotepeque          & 23,104  & 25,027  & 27,076  & 29,573  & 33,649  \\
Intibucá            & 21,543  & 24,266  & 27,105  & 29,370  & 34,357  \\
La Paz              & 19,929  & 22,283  & 24,474  & 26,388  & 30,638  \\
Islas de la Bahía   & 14,970  & 16,872  & 18,644  & 19,528  & 21,559  \\
Gracias a Dios      & 2,021   & 2,266   & 2,650   & 2,930   & 3,727   \\

\caption*{\textit{Fuente:} \acrshort{ine}, en base a datos del Instituto de la Propiedad.}
\end{longtable}
\vspace{-0.5em}

Otra característica visible del tráfico en Comayagua es la diversidad del parque vehicular. Se diferencia 
de grandes ciudades en donde se encuentra un mayor porcentaje de vehículos modernos, en ciudades 
intermedias se observa una mezcla de vehículos antiguos, motocicletas, transporte publico informal, 
vehículos de carga ligera. Esa variabilidad hace que los modelos de tráfico no pueda suponer una 
uniformidad en velocidad, aceleración y comportamiento. Debido a esa heterogeneidad, el comportamiento 
del tráfico puede presentar micro congestiones locales frecuentes. Finalmente, un tema crítico es la 
infraestructura de apoyo limitada. La señalización vial es muy pobre, los sensores, las cámaras, 
carriles exclusivos, aceras, pasos peatonales seguros no están en todos los puntos necesarios, son 
casi inexistentes. Esta deficiencia afecta considerablemente la capacidad de monitoreo y control.

\section{Principales avenidas e intersecciones críticas en Comayagua}
\label{sec:principales-intersecciones-comayagua}

En el contexto local de Comayagua, una de las intersecciones que presenta la mayor saturación vehicular 
es la confluencia donde finaliza el Bulevar 4 Centenario y comienza la Carretera RN-57 
(vía de acceso principal). Esta intersección se ve especialmente sobrecargada al ser alimentada 
simultáneamente por la Calle 7 NO y la 7a CI NO. Según los datos de tráfico \cite{noauthor_google_nodate}, 
esta intersección experimenta picos de congestión vehicular intensa, concentrados en los siguientes 
rangos horarios, los cuales reflejan los movimientos pendulares diarios.

\begin{itemize}
    \item \textbf{9:00 am \- 9:40 am:} Pico matutino tardío.
    \item \textbf{11:40 am \- 3:00 pm:} Período prolongado de mediodía y almuerzo, con incremento de viajes de gestión y salida de jornada escolar.
    \item \textbf{4:20 pm \- 5:40 pm:} Pico vespertino correspondiente a la salida de la jornada laboral.
\end{itemize}
\textit{Fuente: Datos extraídos de Google Maps.}

\clearpage

\begin{figure}[H]
    \centering
    \includegraphics[width=0.9\linewidth]{Figures/4Cent_RN57.png}
    \caption[Interseccion Bulevar 4 Centenario y Carretera RN-57]{Intersección Bulevar 4 Centenario y Carretera RN-57. Fuente: Google Maps \cite{noauthor_google_nodate}}.
    \label{fig:figure-interseccion-4to-centenario-RN57-06}
\end{figure}

Otro punto de congestión relevante en Comayagua se localiza en el Bulevar 4 Centenario, específicamente 
en la intersección alimentada por la 4ª Calle S.O y la antigua CA-5. Es importante destacar que el alto 
nivel de tráfico se presenta primordialmente de lunes a viernes, ya que los fines de semana (sábados y 
domingos) el volumen vehicular experimenta una notable reducción. Durante la semana, los datos de tráfico 
\cite{noauthor_google_nodate} indican que esta zona experimenta picos de congestión significativos, concentrados en 
los siguientes períodos.

\begin{itemize}
    \item \textbf{11:00 am \- 1:00 pm:} Pico de mediodía, relacionado con actividades de almuerzo y gestión.
    \item \textbf{2:00 pm \- 5:50 pm:} Período vespertino de alta congestión, siendo las 4:00 pm el momento en donde se alcanza la tasa máxima de tráfico.
\end{itemize}
\textit{Fuente: Datos extraídos de Google Maps.}

\clearpage


\begin{figure}[H]
    \centering
    \includegraphics[width=0.9\linewidth]{Figures/4Cent_CA5.png}
    \caption[Interseccion Bulevar 4 Centenario con 4a Calle S.O y antigua CA-5]{Intersección Bulevar 4 Centenario con 4a Calle S.O y antigua CA-5. Fuente: Google Maps \cite{noauthor_google_nodate}}
    \label{fig:figure-interseccion-4to-centenario-4a-calle-so-antigua-ca5-06}
\end{figure}

De forma similar a las intersecciones previas, la Calle Cero también registra altos índices de tráfico 
vehicular, especialmente en el cruce donde es alimentada por el Bulevar IV Centenario, la Avenida 1 NO y 
la Avenida 2 NE. Esta intersección se caracteriza por mantener una congestión sostenida durante la mayor 
parte del día, con una reducción apenas perceptible del tráfico fuera de los picos. Los rangos horarios 
de máxima afluencia son:

\begin{itemize}
    \item \textbf{09:00 am \- 11:30 am:} Pico matutino de actividades centrales.
    \item \textbf{04:00 pm \- 06:00 pm:} Pico vespertino, coincidiendo con la finalización de las jornadas laborales.
\end{itemize}
\textit{Fuente: Datos extraídos de Google Maps.}

Es importante destacar que, fuera de estos horarios pico, esta calle se mantiene en un índice de tráfico 
de rango medio, lo que subraya un problema de saturación persistente a lo largo del día.

\clearpage

\begin{figure}[H]
    \centering
    \includegraphics[width=0.9\linewidth]{Figures/Calle0_4Cent.png}
    \caption[Interseccion Calle Cero]{Intersección Calle Cero, Bulevar IV Centenario, Avenida 1 NO y Avenida 2 NE. Fuente: Google Maps \cite{noauthor_google_nodate}}
    \label{fig:figure-interseccion-calle-cero-bulevar-iv-centenario-avenida-1-no-avenida-2-ne-06}
\end{figure}

La antigua carretera de la CA-5 (una carretera principal) sufre de alta congestión durante gran parte del 
día, desde las 10:00 a.m. hasta cerca de las 6:00 p.m., debido a que es el punto de confluencia de otras 
dos carreteras:
\begin{itemize}
    \item RN-68
    \item RV-217 (que está designada para la circulación del transporte interurbano).
\end{itemize}
La presencia del transporte interurbano de manera concentrada en esta vía, sumado al tráfico de las otras 
carreteras que convergen allí, es la causa principal de que pase la "mayor parte del tiempo 
congestionada".

\clearpage

\begin{figure}[H]
    \centering
    \includegraphics[width=0.9\linewidth]{Figures/CA5_RV217.png}
    \caption[Interseccion CA-5, Carretera RV-217 y RN-68]{Intersección CA-5, Carretera RV-217 y RN-68. Fuente: Google Maps \cite{noauthor_google_nodate}}
    \label{fig:figure-interseccion-ca5-estadio-06}
\end{figure}


\section{Relevancia del uso de inteligencia artificial en la ciudad}
\label{sec:relevancia-ia-comayagua}

La ciudad de Comayagua, ubicada estratégicamente en el centro de Honduras, atraviesa un acelerado crecimiento vehicular y urbano derivado del desarrollo de infraestructura 
clave como el Aeropuerto Internacional de Palmerola y la modernización de la carretera CA-5. Este proceso ha impulsado la actividad económica local, pero también ha generado 
una creciente presión sobre la red vial existente, evidenciada en el aumento de los tiempos de desplazamiento y la saturación de las principales rutas de circulación.

Ante este contexto, el análisis de la movilidad urbana mediante modelos basados en teoría de grafos se presenta como una alternativa rigurosa y accesible para comprender el 
comportamiento estructural del tráfico. La representación de la ciudad como un grafo donde las intersecciones se modelan como nodos y las vías como aristas permite estudiar 
el nivel de conectividad, identificar rutas críticas, evaluar vulnerabilidades y estimar cómo los flujos vehiculares se distribuyen a través de la red. Este enfoque proporciona 
un marco sistemático para evaluar cuellos de botella, proponer rediseños de intersecciones y anticipar cómo el crecimiento urbano impactará sobre la infraestructura existente.

Además, el uso de herramientas de simulación basadas en redes, como SUMO, fortalece la capacidad analítica al permitir modelar escenarios realistas de tránsito, evaluar cambios 
en la infraestructura vial y examinar su impacto antes de ejecutarlos físicamente. En un país donde los sistemas avanzados de monitoreo vial aún son limitados, estos métodos 
representan una oportunidad para impulsar estrategias de movilidad más eficientes, sostenibles y fundamentadas en análisis cuantitativos.

En Honduras, si bien las iniciativas tecnológicas aplicadas a la movilidad aún se encuentran en una etapa temprana, el \textbf{Plan de Movilidad} Urbana constituye un avance 
significativo, ya que establece lineamientos para la digitalización, levantamiento y sistematización de datos de tráfico vehicular y movilidad peatonal \cite{noauthor_plan_nodate}. 
Este tipo de información es fundamental para representar la infraestructura vial mediante modelos basados en grafos, permitiendo analizar la estructura de la red, identificar 
puntos críticos y proponer mejoras orientadas a la eficiencia y sostenibilidad del transporte urbano.

Al adoptar la teoría de grafos como eje metodológico, Comayagua puede avanzar hacia una gestión urbana más técnica y basada en evidencia, posicionándose como un referente 
nacional y regional en la planificación moderna del transporte.
