\chapter{Investigaciones previas relacionadas}
\label{chap:investigaciones-previas}

\insertminitoc
\parindent0pt

\section{Antecedentes y estado actual de la investigación}

En la última década, la gestión académica ha transitado de modelos administrativos estáticos a enfoques dinámicos basados en datos. La literatura reciente coincide en que la sostenibilidad de las instituciones de educación superior depende ahora de integrar tecnología en la planificación operativa, más allá de su oferta pedagógica. No obstante, la adopción de estas herramientas varía según el contexto geográfico y cultural.

Al contrastar las exigencias del entorno global con la realidad operativa latinoamericana, se evidencia una brecha crítica en la gestión de recursos. Mientras que Shenkoya y Kim\cite{shenkoya_sustainability_2023}, establecieron que la transformación digital es un imperativo de supervivencia para la sostenibilidad en la Cuarta Revolución Industrial, Gallegos Macías et al\cite{gallegos_macias_sistemas_2023}, demostraron que, en la práctica regional, esta transición es deficiente. A pesar de que las instituciones invierten en infraestructura tecnológica, Gallegos Macías et al. diagnosticaron que los Sistemas de Información Estratégica (SIE) funcionan mayoritariamente como repositorios de datos aislados. Esta desconexión confirma que el problema no es la falta de herramientas digitales, sino la ausencia de una cultura organizacional que alinee los datos operativos con la visión estratégica de sostenibilidad propuesta por Shenkoya y Kim\cite{shenkoya_sustainability_2023}.

En el ámbito metodológico, la insuficiencia de los modelos estadísticos tradicionales para predecir la demanda fue validada convergentemente por Shao et al.\cite{shao_machine_2021} y Shilbayeh y Abonamah\cite{shilbayeh_predicting_2021}. Ambos estudios coincidieron en que la regresión lineal y logística fallan al intentar capturar la complejidad del comportamiento estudiantil actual. Shao et al.\cite{shao_machine_2021} ,comprobaron que algoritmos de ensamble como Random Forest eran necesarios para detectar interacciones no lineales entre el historial académico y la elección de carrera, una capacidad que los modelos clásicos no poseen. Por su parte, Shilbayeh y Abonamah\cite{shilbayeh_predicting_2021}, corroboraron esta superioridad técnica, demostrando que al sustituir los métodos probabilísticos por árboles de decisión (J48), la precisión en la clasificación de estudiantes en riesgo aumentó significativamente, permitiendo una planificación proactiva y no reactiva.

La evolución del estado actual de la investigación también ha redefinido qué datos son relevantes para la planificación. Se superó la visión de que el rendimiento académico es el único predictor de la matrícula. Shao et al.\cite{shao_machine_2021} jerarquizaron la importancia de las variables (feature importance), revelando que factores demográficos específicos tienen un peso predictivo superior al 35\%. Este hallazgo se complementa con la investigación de Shilbayeh y Abonamah\cite{shilbayeh_predicting_2021}, quienes integraron variables financieras en sus modelos, descubriendo una correlación directa (98.6\% de precisión) entre el estatus económico y la deserción. En conjunto, estos autores establecieron que un sistema de predicción robusto debe ser multidimensional, integrando datos académicos, demográficos y financieros para reducir la incertidumbre en la oferta de cupos.

Un avance fundamental en los antecedentes de la optimización académica es el cambio de paradigma en la visualización del plan de estudios. Tradicionalmente, la oferta se planificaba como una lista lineal de asignaturas, un enfoque que Stavrinides y Zuev\cite{stavrinides_course-prerequisite_2023}, desafiaron al proponer el modelado mediante Redes de Prerrequisitos de Cursos (CPNs). Al aplicar la teoría de grafos, estos autores demostraron que la complejidad de la gestión no reside solo en la cantidad de alumnos, sino en la topología de la malla curricular.

La aplicación de métricas de centralidad permitió cuantificar fenómenos que antes solo se gestionaban intuitivamente. Stavrinides y Zuev\cite{stavrinides_course-prerequisite_2023}, introdujeron el uso de la ``Centralidad de Intermediación'' (Betweenness Centrality) para detectar asignaturas que actúan como puentes críticos en la red de conocimiento.

Finalmente, la comprensión de la jerarquía curricular se consolidó mediante la estratificación topológica. Stavrinides y Zuev\cite{stavrinides_course-prerequisite_2023}, establecieron que los planes de estudio poseen una estructura de niveles de dependencia que dicta el flujo natural de la demanda.


\section{Estado del arte}

En el estado actual del conocimiento, la optimización de la oferta académica supera la estadística tradicional e incorpora computación evolutiva y análisis de sistemas complejos. El estado del arte aborda problemas de asignación intratables para humanos, integrando infraestructura para modalidades híbridas y el comportamiento digital de estudiantes, que redefinen la planificación eficiente.

La resolución del Problema de Horarios Universitarios (UCTP) ha alcanzado un nivel de sofisticación que separa definitivamente la gestión manual de la automatizada. Abdipoor et al.\cite{abdipoor_meta-heuristic_2023}, definieron este problema como NP-Hard, estableciendo que la única vía viable para su resolución es el uso de metaheurísticas híbridas.

Una dimensión emergente en el estado del arte es la validación logística de la educación híbrida. Guadalupe Beltrán et al.\cite{guadalupe_beltran_desafios_2025} aportaron evidencia reciente sobre cómo la modalidad mixta ha introducido nuevas restricciones de infraestructura. Su estudio reveló que el 45\% de los fallos en la implementación docente no se deben a la pedagogía, sino a la asignación de aulas físicamente incompatibles con la transmisión digital.

Finalmente, la frontera de la predicción de demanda se ha expandido hacia el análisis de datos no estructurados externos. Li et al.\cite{li_exploring_2025} demostraron mediante modelos de ecuaciones estructurales que el comportamiento en redes sociales (eWOM) actúa como un indicador adelantado de la inscripción, estableciendo un nuevo estándar para los sistemas de planificación modernos.

A modo de síntesis, la revisión crítica de la literatura permite constatar que la gestión de la oferta académica ha dejado de ser un problema logístico lineal para constituirse como un desafío multidimensional de ciencia de datos. La evidencia analizada demuestra que, si bien existen herramientas teóricas potentes desde los algoritmos de Random Forest validados por Shao et al.\cite{shao_machine_2021}, hasta el análisis de grafos curriculares de Stavrinides y Zuev\cite{stavrinides_course-prerequisite_2023}, su aplicación práctica enfrenta la necesidad de integrar variables heterogéneas que los modelos tradicionales ignoraban. Se concluye que el estado del arte actual exige el desarrollo de sistemas holísticos que no solo resuelvan la complejidad combinatoria definida por Abdipoor et al.\cite{abdipoor_meta-heuristic_2023}, sino que incorporen simultáneamente las restricciones de infraestructura híbrida y los indicadores de demanda digital externa, cerrando así la brecha existente entre la capacidad computacional avanzada y la realidad operativa de las instituciones educativas.
