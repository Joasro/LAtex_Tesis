\chapter[Introducci\'on]{Introducción}
\label{cp:introduction}

\insertminitoc
\parindent0pt

\section{Planteamiento del Problema}
A nivel mundial, la educación superior atraviesa una crisis de "masificación" sin precedentes. Según datos de la UNESCO\cite{noauthor_higher_2026}, la matrícula universitaria global se ha triplicado en los últimos años, tomando de referencia desde el año 2000 hasta datos más recientes en 2024, pasando de 100 millones a más de 264 millones de estudiantes, lo que ha generado una presión insostenible sobre la infraestructura y los recursos docentes de las instituciones públicas. Este crecimiento desmedido, si no se gestiona con herramientas de planificación estratégica, deriva en la saturación de aulas y la incapacidad de las universidades para ofertar los cupos necesarios que la población estudiantil demanda.

La ineficiencia en la gestión de la oferta académica tiene consecuencias directas en el éxito estudiantil. El Banco Mundial\cite{BancoMundial2021} señala que en América Latina, aproximadamente el 46\% de los estudiantes que ingresan a la educación superior logran graduarse, y de ellos, una gran mayoría tarda hasta un 40\% más del tiempo estipulado en su plan de estudios. Este fenómeno, conocido como el rezago académico, es alimentado en gran medida por la falta de disponibilidad de asignaturas críticas, lo que obliga a los estudiantes a extender su permanencia en la universidad, elevando los costos operativos para el Estado y retrasando su inserción en el mercado laboral.

En el contexto regional, la Organización de Estados Iberoamericanos (OEI) ha identificado debilidades estructurales en la gestión universitaria. Los informes de diagnóstico revelan un rezago tecnológico significativo, donde el ecosistema digital de la región presenta apenas un 50\% de desarrollo en comparación con el 80\% alcanzado en América del Norte\cite{marquina_informe_2022}. Esta disparidad pone de manifiesto una infraestructura insuficiente para afrontar los desafíos de la planificación moderna. A esta situación se suma una desconexión en la pertinencia académica; estudios de competitividad señalan la existencia de una ``importante y creciente brecha'' entre la formación ofertada y los requerimientos reales del entorno\cite{OEI2020}, fenómeno que deriva en la subutilización del capital humano y obstaculiza la inserción laboral de los egresados.

Por su parte, la gestión administrativa en las universidades públicas hondureñas enfrenta limitaciones técnicas que restringen su capacidad de optimización. La implementación de tecnologías de la información se ha circunscrito mayoritariamente a procesos operativos básicos, dejando de lado el uso de herramientas analíticas avanzadas para la planificación estratégica~\cite{BancoMundial2021}. En consecuencia, la toma de decisiones sobre la programación de secciones académicas continúa sustentándose en criterios empíricos y tradicionales, lo que conlleva a ignorar patrones subyacentes en los datos históricos institucionales. Esta carencia de mecanismos de predicción y análisis compromete la eficiencia operativa y dificulta el aseguramiento de una educación inclusiva y equitativa, metas fundamentales del \acrfull{ods} N.º 4 de la Agenda 2030\cite{moran_educacion_nodate}.


\section{Objetivos de la investigación}

\subsection{Objetivo General}
Optimizar la planificación de la oferta académica en instituciones de educación mediante el desarrollo de un modelo de inteligencia artificial que soporte la toma de decisiones para una distribución eficiente de los recursos institucionales.

\subsection{Objetivos Específicos}

\begin{itemize}
    \item Analizar datos históricos de cargas académicas y matrícula para identificar patrones de demanda en distintas asignaturas.
    \item Examinar y priorizar las variables académicas y normativas que influyen en la apertura de secciones, tales como mínimos de estudiantes y prioridades académicas.
    \item Diseñar un modelo predictivo que permita estimar la demanda de secciones académicas en futuros períodos.
    \item Evaluar el desempeño del modelo a través de un conjunto de datos de prueba a pequeña escala.
\end{itemize}


\section{Preguntas de investigación}

\begin{itemize}
    \item ¿Qué patrones de demanda pueden identificarse a partir del análisis de datos históricos de matrícula y cargas académicas?
    \item ¿Cuáles son las variables académicas y normativas que influyen con mayor peso en la apertura de secciones académicas?
    \item ¿Cómo puede diseñarse un modelo predictivo basado en aprendizaje automático que permita estimar la demanda futura de secciones académicas?
    \item ¿Qué nivel de desempeño presenta el modelo predictivo al ser evaluado con datos históricos de prueba?
\end{itemize}


\section{Justificación de la investigación}

\subsection{Justificación Teórica y Técnica}

La presente investigación se justifica en la necesidad de superar las limitaciones de los modelos estadísticos tradicionales utilizados en la gestión educativa. La literatura reciente demuestra que métodos como la regresión lineal o logística son insuficientes para capturar la complejidad no lineal del comportamiento estudiantil actual \cite{shao_machine_2021}. Al implementar algoritmos de aprendizaje automático, específicamente Random Forest, este estudio busca llenar un vacío metodológico, validando herramientas que permiten jerarquizar la importancia de variables demográficas y académicas con una precisión superior al análisis probabilístico estándar \cite{shao_machine_2021}.

Asimismo, se aborda la programación de horarios como un problema de optimización combinatoria clasificado como \acrfull{np}-Hard \cite{abdipoor_meta-heuristic_2023}. La investigación aporta valor teórico al demostrar que la gestión de restricciones ``duras'' y ``blandas'' en la asignación de recursos no es eficiente con planificación manual, justificando el uso de metaheurísticas y modelos predictivos como la única vía para garantizar soluciones factibles en espacios de búsqueda de alta complejidad \cite{abdipoor_meta-heuristic_2023}.

\subsection{Justificación Práctica y Administrativa}

Desde una perspectiva operativa, el proyecto responde a la desconexión crítica diagnosticada entre la infraestructura tecnológica y la toma de decisiones estratégicas en las universidades latinoamericanas \cite{gallegos_macias_sistemas_2023}. A pesar de que las instituciones generan grandes volúmenes de datos, la falta de una cultura de análisis ha convertido a los sistemas de información en simples repositorios pasivos \cite{gallegos_macias_sistemas_2023}. Esta propuesta ofrece una solución práctica mediante un \acrfull{dss} que transforma datos históricos aislados en inteligencia accionable, permitiendo a las autoridades anticipar la demanda real de cupos y evitar la gestión reactiva.

Además, esta investigación responde a los retos de la educación híbrida. Evidencia reciente del año 2025 indica que una proporción significativa de los docentes (aproximadamente un tercio) percibe que los recursos tecnológicos actuales son insuficientes para soportar adecuadamente la modalidad mixta \cite{guadalupe_beltran_desafios_2025}. El sistema se justifica al integrar variables de infraestructura física y digital en la planificación, asegurando que la asignación de espacios responda a los requerimientos técnicos actuales y mitigue las barreras operativas detectadas.

\subsection{Justificación Económica}

La optimización de la oferta académica tiene un impacto directo en la sostenibilidad financiera de la institución. La incapacidad para predecir con exactitud las tasas de inscripción genera costos administrativos innecesarios, derivados de la apertura de secciones con baja demanda o la saturación de otras que requieren contratación docente de emergencia \cite{shao_machine_2021}. Al minimizar el error de predicción, esta investigación contribuye a la eficiencia presupuestaria, permitiendo una asignación de recursos (humanos y físicos) ajustada a la demanda real, lo cual es un imperativo para la sostenibilidad en el contexto de la Cuarta Revolución Industrial \cite{shenkoya_sustainability_2023}.

\subsection{Justificación Social}

Finalmente, el impacto más significativo recae sobre el éxito estudiantil. La falta de disponibilidad de cupos en asignaturas críticas, conocidas como ``cuellos de botella'', interrumpe el flujo curricular y es un factor determinante en el rezago académico \cite{stavrinides_course-prerequisite_2023}. Al utilizar análisis de redes para identificar estos nodos críticos y predecir el riesgo de deserción asociado a factores financieros y académicos con una precisión cercana al 89\% \cite{shilbayeh_predicting_2021}, el sistema propuesto no solo optimiza procesos, sino que favorece la continuidad educativa. Esto garantiza que los estudiantes puedan avanzar en su malla curricular sin bloqueos administrativos, reduciendo las tasas de abandono y facilitando su inserción oportuna en el mercado laboral.

\section{Alcances y limitaciones}

Esta investigación comprende el diseño y desarrollo de un prototipo funcional de plataforma digital basada en inteligencia artificial para predecir la demanda académica en UNAH-Comayagua. El sistema procesará registros de intención de matrícula y datos históricos mediante algoritmos de regresión y clasificación. El alcance incluye la creación de perfiles de usuario, un módulo de análisis predictivo y un dashboard de visualización para autoridades. Entre las limitaciones, se reconoce que la variabilidad en la calidad de los datos históricos puede afectar la precisión inicial, y que las pruebas piloto se limitarán a una muestra representativa de X estudiantes debido a restricciones temporales.

\begin{itemize}
    \item \textbf{Viabilidad Técnica}: El proyecto se desarrollará utilizando el lenguaje de programación Python, empleando librerías especializadas como Pandas y Scikit-Learn. La gestión de datos se realizará mediante el motor de base de datos MySQL. La interfaz de usuario será implementada con el framework Streamlit y el código fuente será auditado mediante un repositorio en GitHub.
    \item \textbf{Viabilidad Económica y Operativa}: La investigación cuenta con la viabilidad operativa al disponer del apoyo de la Jefatura de Departamento de UNAH-Comayagua para el acceso a la información. Es económicamente viable al basarse en tecnologías de código abierto, no requiriendo inversión financiera directa por parte de la institución.
\end{itemize}
