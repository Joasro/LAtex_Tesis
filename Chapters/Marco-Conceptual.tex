\chapter[Marco Conceptual]{Marco Conceptual}
\label{ch:marco-conceptual}

\clearpage

\section{Glosario conceptual del estudio}

\begin{itemize}
    \item[] \textbf{Algoritmo Genético (GA):} Técnica de resolución de problemas en inteligencia artificial inspirada en la evolución biológica (selección, cruce y mutación). Se utiliza para encontrar soluciones óptimas en escenarios con millones de combinaciones posibles, como la creación automatizada de horarios sin choques de clases. \\
    \item[] \textbf{API (Interfaz de Programación de Aplicaciones):} Conjunto de reglas y protocolos tecnológicos que permite que dos sistemas informáticos diferentes se comuniquen entre sí de forma segura y automatizada. En el contexto educativo, permite que los datos del expediente del estudiante viajen al motor de inteligencia artificial en tiempo real. \\
    \item[] \textbf{API RESTful:} Un tipo específico de arquitectura de API diseñada para ser ligera, rápida y altamente escalable en entornos web. Es el estándar de la industria para el consumo y envío de datos entre servidores y aplicaciones móviles o web. \\
    \item[] \textbf{Arquitectura de Microservicios:} Método de diseño de software donde un sistema complejo se divide en componentes pequeños, independientes y especializados (servicios). Si una parte del sistema falla o recibe mucha demanda, el resto del sistema sigue funcionando sin colapsar. \\
    \item[] \textbf{Big Data (Datos Masivos):} Conjuntos de datos cuyo volumen, complejidad y velocidad de crecimiento son tan grandes que los programas informáticos de procesamiento tradicional no logran capturarlos, gestionarlos ni analizarlos de forma eficiente. \\
    \item[] \textbf{Caja Negra (Black Box):} Término utilizado para describir modelos de Inteligencia Artificial donde se conocen los datos que entran y los resultados que salen, pero el proceso lógico interno que la máquina usó para llegar a esa conclusión es incomprensible para el ser humano. \\
    \item[] \textbf{Clustering (Agrupamiento / K-medoids):} Técnica de aprendizaje automático que agrupa automáticamente conjuntos de datos no etiquetados en categorías basadas en sus similitudes. Sirve, por ejemplo, para agrupar estudiantes con perfiles de riesgo académico similares. \\
    \item[] \textbf{DSS (Decision Support System / Sistema de Soporte a la Decisión):} Sistema informático interactivo diseñado para ayudar a los administradores a tomar decisiones estratégicas. En lugar de automatizar la decisión final, el sistema analiza grandes volúmenes de datos y presenta escenarios probables al usuario humano. \\
    \item[] \textbf{ETL (Extracción, Transformación y Carga / Pipelines):} Proceso automatizado de ingeniería de datos. Consiste en extraer datos crudos de múltiples fuentes, limpiarlos/estandarizarlos (transformación) y guardarlos en una base de datos moderna (carga) lista para la Inteligencia Artificial. \\
    \item[] \textbf{Grafo Dirigido Acíclico (DAG):} Representación matemática en forma de red donde la información fluye en una sola dirección y nunca forma un ciclo cerrado. Se utiliza para mapear mallas curriculares, impidiendo que el estudiante avance en círculos. \\
    \item[] \textbf{Heurística y Metaheurística:} Métodos matemáticos y algoritmos que buscan encontrar soluciones ``suficientemente buenas'' y rápidas a problemas que son demasiado complejos para calcular una solución perfecta en un tiempo razonable. \\
    \item[] \textbf{IMS (Information Management System / Sistema de Gestión de Información):} Plataforma tecnológica centralizada que utilizan las instituciones para gestionar sus procesos operativos, académicos y administrativos (comúnmente conocido como el portal de registro o ERP). \\
    \item[] \textbf{Inteligencia Artificial Explicable (XAI):} Conjunto de técnicas diseñadas para traducir el razonamiento de un algoritmo a un formato visual que un humano pueda entender, generando confianza al permitir saber exactamente por qué la máquina recomendó una acción. \\
    \item[] \textbf{Kubernetes (Orquestación de Contenedores):} Sistema tecnológico que administra y ajusta automáticamente la capacidad de las aplicaciones de software según la demanda, asignando temporalmente más potencia de cálculo para evitar caídas del sistema. \\
    \item[] \textbf{Machine Learning (Aprendizaje Automático):} Rama de la inteligencia artificial que permite a los sistemas informáticos aprender a identificar patrones y mejorar su rendimiento a partir de datos históricos, sin necesidad de que un humano los programe con reglas explícitas. \\
    \item[] \textbf{Minería de Datos Educativa (EDM):} Proceso de descubrir patrones ocultos, correlaciones y anomalías dentro de grandes bases de datos estudiantiles, con el fin de predecir el comportamiento del alumnado y mejorar las políticas de administración institucional. \\
    \item[] \textbf{Modelo de Regresión (Regresión Polinómica):} Técnica estadística y de aprendizaje automático que analiza la relación entre múltiples variables para predecir un valor numérico continuo, adaptándose a comportamientos que no son líneas rectas predecibles. \\
    \item[] \textbf{Problema NP-Duro (NP-Hard):} Categoría de problemas matemáticos de altísima complejidad donde el número de combinaciones posibles es tan masivamente grande que calcular todas las opciones exhaustivamente colapsaría a una computadora convencional. \\
    \item[] \textbf{Random Forest (Bosque Aleatorio):} Poderoso algoritmo de predicción que funciona creando múltiples árboles de decisión durante su entrenamiento. Para dar un resultado final, promedia las conclusiones de todos esos árboles, haciéndolo altamente preciso. \\
    \item[] \textbf{SLM (Small Language Models / Modelos de Lenguaje Pequeños):} Sistemas de inteligencia artificial capaces de comprender y procesar texto humano de manera eficiente, utilizados para extraer datos útiles de documentos desordenados sin requerir supercomputadoras masivas. \\
    \item[] \textbf{Suavizado Exponencial (Series Temporales):} Método matemático de pronóstico que analiza datos históricos a lo largo del tiempo, dándole mayor peso de importancia a los datos más recientes para predecir tendencias futuras. \\
    \item[] \textbf{UCTP (University Course Timetabling Problem):} El desafío logístico clásico en ciencias de la computación que busca ubicar un conjunto de eventos académicos en franjas horarias y aulas limitadas, sin que profesores o estudiantes tengan conflictos de horario.
\end{itemize}