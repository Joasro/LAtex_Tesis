\chapter[Marco Conceptual]{Marco Conceptual}
\label{ch:marco-conceptual}

\clearpage

\section{Glosario conceptual del estudio}

\begin{itemize}
    \item[] \textbf{Movilidad Urbana:} Conjunto de condiciones, infraestructuras y dinámicas que permiten el desplazamiento eficiente de personas y bienes dentro de una ciudad, considerando accesibilidad, seguridad, tiempos de viaje y calidad de vida.\\
    \item[] \textbf{Congestionamiento Vehicular:} Situación en la que la demanda de circulación supera la capacidad de la infraestructura vial, produciendo reducción de velocidad, aumento de densidad y retrasos generalizados. \\
    \item[] \textbf{Intersección Crítica:} Punto de la red vial donde convergen varios flujos y se presenta un alto riesgo de saturación debido a su importancia dentro de la movilidad urbana. \\
    \item[] \textbf{Teoría de Grafos:} Rama de las matemáticas que estudia estructuras formadas por nodos y aristas, utilizada para modelar redes viales, flujos de tráfico y rutas óptimas. \\
    \item[] \textbf{Nodo:} Punto que representa una intersección, cruce o conexión dentro de un grafo, equivalente a un punto de decisión dentro de la red vial. \\
    \item[] \textbf{Arista:} Conexión entre dos nodos en un grafo; en la red vial representa un tramo de calle o carretera por donde circulan los vehículos. \\
    \item[] \textbf{Sistemas Adaptativos de Control de Tráfico (ATSC):} Tecnologías que modifican automáticamente la fase y duración de los semáforos según datos en tiempo real, mejorando la fluidez del tráfico. \\
    \item[] \textbf{Grafo Dirigido:} Tipo de grafo en el que las aristas tienen una dirección específica, útil para modelar calles de un solo sentido o flujos vehiculares definidos. \\
    \item[] \textbf{Grafo Ponderado:} Estructura de grafo en la que cada arista tiene un peso asociado, como distancia, tiempo o costo, permitiendo analizar rutas y eficiencia vial. \\
    \item[] \textbf{Camino Mínimo:} Ruta de menor distancia o costo dentro de un grafo entre dos puntos, calculada mediante algoritmos como Dijkstra o A*. \\
    \item[] \textbf{Centralidad de Intermediación:} Métrica en teoría de grafos que indica qué tan frecuentemente un nodo aparece en las rutas más cortas. En tráfico identifica intersecciones críticas. \\
    \item[] \textbf{Cuello de Botella:} Zona de la red vial donde la capacidad es insuficiente para soportar la demanda, generando acumulación de vehículos y lentitud en el flujo. \\
    \item[] \textbf{Simulación de Tráfico:} Proceso computacional que reproduce el comportamiento de vehículos en una red vial para analizar escenarios, predecir congestión y evaluar soluciones. \\
    \item[] \textbf{SUMO (Simulation of Urban Mobility):} Herramienta de simulación de tráfico microscópico que permite modelar redes viales, controlar semáforos, evaluar escenarios y analizar el comportamiento vehicular.
\end{itemize}


