\chapter[Marco Legal]{Marco Legal}
\label{ch:marco-legal}

\insertminitoc
\parindent0pt

\section{Normativa legal y estándares técnicos}
A nivel internacional, existen diversos marcos regulatorios y estándares que orientan el diseño, 
la interoperabilidad y la seguridad de los \acrshort{its} y de las 
aplicaciones basadas en inteligencia artificial. Algunos de los más relevantes para esta investigación 
son:\\

\begin{itemize}
    \item \textbf{ISO 14813 (Intelligent Transport Systems \- ITS Reference Model Architecture):}\\
    Define una arquitectura de referencia para los sistemas de transporte inteligente, estableciendo 
    una estructura común para la integración de componentes de hardware, software y comunicación dentro 
    de los \acrshort{its}. Este estándar orienta cómo deben organizarse los subsistemas y servicios para garantizar 
    la interoperabilidad entre distintos entornos urbanos.
    \item \textbf{ISO 39001:2012 (Road Traffic Safety Management Systems):}\\
    Se fundamenta en la idea de que la responsabilidad es un asunto de todos: cada entidad con injerencia 
    en el ámbito del transporte terrestre debe velar por la protección en las carreteras \cite{lie_iso_2022}.
    \item \textbf{IEEE 1609 \- Wireless Access in Vehicular Environments (WAVE):}\\
    Se establece una estructura técnica y se normaliza una serie de funciones y puntos de conexión 
    que facilitan la intercomunicación inalámbrica protegida y la entrada física dentro del automóvil \cite{noauthor_pdf_2025}.
    \item \textbf{IEEE 2846-2022 \- Assumptions for Safety-Related Models in Automated Vehicle Behavior:}\\
    El estándar IEEE establece las premisas básicas sobre el movimiento de los usuarios viales que un 
    modelo de seguridad en un Sistema de Conducción Automatizada debe tener en cuenta sin falta \cite{ieee_vt_literature_2022}.
\end{itemize}

\section{Normativa regional (Latinoamérica)}
En América Latina, las políticas sobre movilidad inteligente y tecnologías de IA aún se encuentran 
en proceso de consolidación, pero existen lineamientos relevantes emitidos por organismos regionales 
y multilaterales:

\begin{itemize}
    \item \textbf{Banco Interamericano de Desarrollo (BID):} aborda la necesidad crítica de mejorar la 
    movilidad y el acceso al transporte para las poblaciones vulnerables en América Latina y el Caribe. 
    La publicación define la movilidad universal como la capacidad de transportarse sin peligros y con 
    información adecuada, y examina cómo los ITS a través de la tecnología, son herramientas esenciales 
    para hacer que el transporte público y privado sea más accesible, seguro y confiable \cite{granada_sistemas_2018}.
    \item \textbf{Comisión Económica para América Latina y el Caribe (CEPAL):} proporciona un panorama 
    detallado sobre cómo la transformación digital y la IA pueden ayudar a América Latina y el Caribe a 
    superar sus principales obstáculos de desarrollo, identificados como la baja productividad, la alta 
    desigualdad y la debilidad institucional \cite{noauthor_superar_nodate}.
\end{itemize} 

\section{Marco Legal Nacional (Honduras)}
En Honduras, aunque no existe una ley específica que regule directamente los sistemas inteligentes de 
transporte o la inteligencia artificial, sí hay marcos normativos generales que inciden en su 
desarrollo y aplicación:

\begin{itemize}
    \item \textbf{Ley de Tránsito:} Regula la circulación vehicular, las competencias de la Dirección 
    Nacional de Vialidad y Transporte (DNVT) y la gestión de la seguridad vial.
    \item \textbf{Ley de Telecomunicaciones (Decreto No. 185-95):} Regula el uso de redes y sistemas de 
    comunicación, lo que incluye la transmisión de datos vehiculares y la interconexión entre \acrshort{its} 
    en caso de que se incorporen futuras plataformas conectadas \cite{noauthor_ley_nodate}.
    \item \textbf{Agenda Digital Honduras 2030:} Plan estratégico a nivel nacional que le da un empuje 
    al uso de tecnologías de punta, con un enfoque importante en optimizar la infraestructura, incluyendo 
    la red vial del país. Este esquema fomenta iniciativas de innovación tecnológica enfocadas en el 
    desarrollo sostenible \cite{noauthor_honduras_nodate}.
\end{itemize}

\section{Consideraciones éticas y de seguridad de la información}
Todo desarrollo basado en inteligencia artificial debe regirse por principios éticos y de seguridad. 
En este sentido, se tomaron en cuenta los lineamientos del “Ethics Guidelines for Trustworthy AI” 
\cite{smuha_eu_2019} de la Comisión Europea (2019), los cuales establecen que la IA debe ser:

\begin{itemize}
    \item Legal (cumpliendo las leyes y regulaciones vigentes),
    \item Ética (respetando los derechos fundamentales y valores humanos), y
    \item Robusta (técnica y socialmente).
\end{itemize}

Asegurar el apego a estas directrices es crucial para que el esquema de simulación y estudio de 
tráfico que planteamos resulte claro, comprensible y fiable, inclusive cuando los datos empleados 
sean artificiales o meramente una estimación.