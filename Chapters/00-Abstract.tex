\thispagestyle{plain}
\chapter*{Resumen}

Esta tesis se centra en la evaluación del flujo vehicular en las principales intersecciones de la ciudad de Comayagua que experimentan congestión. Inicialmente, se aplicó la observación directa como técnica para estimar el volumen de tráfico que circula por estas zonas críticas.

Posteriormente, se realizó un análisis de la red vial mediante la teoría de grafos, utilizando las librerías de Python OSMx y NetworkX. Buscando identificar las intersecciones (nodos) con una alta puntuación de centralidad, que son esenciales por conectar la mayoría de las rutas más cortas. Una vez definidas las calles críticas, se ejecutó una simulación de tráfico con la herramienta SUMO bajo tres escenarios: el flujo base, un incremento del $30\%$, y una duplicación del flujo vehicular. Además, se comparó el desempeño del tráfico en el nodo crítico al emplear un semáforo contra la instalación de una rotonda.

Los resultados demostraron una mejora significativa en la movilidad, evidenciada por un aumento en la velocidad media, una reducción del tiempo perdido, y un incremento en las paradas por vehículo, sugiriendo una mayor fluidez. Validando la factibilidad de emplear la teoría de grafos y herramientas de simulación para proponer estrategias efectivas de descongestión vehicular y fundamentar la toma de decisiones.

\keywordses{Tráfico vehicular, Teoría de grafos, Simulación, Congestionamiento}

\MediaOptionLogicBlank

\pdfbookmark[1]{Abstract}{abstract}
\chapter*{Abstract}

This thesis focuses on evaluating vehicular flow at the main intersections of the city of Comayagua that experience congestion. Initially, direct observation was applied as a technique to estimate the traffic volume circulating through these critical areas.

Subsequently, a road network analysis was conducted using graph theory, employing the Python libraries OSMnx and NetworkX. The objective was to identify intersections (nodes) with a high centrality score, which are essential because they connect most of the shortest routes. Once the critical streets were defined, a traffic simulation was executed using the SUMO tool under three scenarios: the baseline flow, a 30% increase, and a doubling of vehicular flow. In addition, the traffic performance at the critical node was compared when using a traffic light versus the installation of a roundabout.

The results demonstrated a significant improvement in mobility, evidenced by an increase in average speed, a reduction in lost time, and an increase in stops per vehicle, suggesting greater fluidity. This validates the feasibility of using graph theory and simulation tools to propose effective vehicular decongestion strategies and support decision-making.

\keywordsen{Vehicular Traffic, Graph Theory, Simulation, Congestion}

\MediaOptionLogicBlank