\chapter[Anexos]{Anexos}
\label{chap:anexos}
\addcontentsline{toc}{chapter}{Anexos}


\begin{tcolorbox}[colback=gray!5!white, colframe=gray!60!black, title=Anexo A \- Proceso de configuración de SUMO]
	En este anexo se presentan unas fotografías del proceso de configuración de SUMO.
    En la imagen se observa como se esta trabajando con la herramienta de edición de redes de SUMO (NETEDIT) para la
    creación de la red vial correspondiente al área de estudio.
\end{tcolorbox}

\begin{figure}[H]
    \centering
    \includegraphics[width=0.8\textwidth]{Figures/Anexos/yo_!.jpg}
    \caption{Interfaz de SUMO NETEDIT para la edición de redes viales}
    \label{fig:anexo-sumo-netedit}
\end{figure}

\begin{tcolorbox}[colback=gray!5!white, colframe=gray!60!black, title=Anexo B \- Equipo de cómputo utilizado]
	En este anexo se muestra el equipo de cómputo utilizado para la realización de las simulaciones
    vehiculares y análisis de datos presentados en este trabajo.
\end{tcolorbox}

\begin{figure}[H]
    \centering
    \includegraphics[width=0.8\textwidth]{Figures/Anexos/pc.jpg}
    \caption{Equipo de cómputo utilizado para la realización de las simulaciones y análisis de datos}
    \label{fig:anexo-equipo-computo}
\end{figure}

\clearpage

\begin{tcolorbox}[colback=gray!5!white, colframe=gray!60!black, title=Anexo C \- Código en pyhton para la generación del grafos]
	En este anexo se presenta un fragmento de código utilizado para la generación y análisis del grafo
    vial del área de estudio utilizando la librería OSMNX y NetworkX.
\end{tcolorbox}

\begin{listing}[H]
    \caption{Fragmento de código en python para la generación y análisis del grafo}
    \label{cod:anexo-grafo}
    \begin{minted}{python}
        G2 = ox.consolidate_intersections(ox.project_graph(G_simplified), tolerance=15)

        for u, v, a in G2.edges(data=True):
        if 'maxspeed' not in a:
            a['maxspeed'] = 50
        if isinstance(a['maxspeed'], list):
            a['maxspeed'] = min(a['maxspeed'])
        elif isinstance(a['maxspeed'], dict):
            print(a)

        if isinstance(a['maxspeed'], str):
            a['maxspeed'] = float(a['maxspeed'])

        a['travel_time'] = a['length'] / a['maxspeed']

        G2[u][v][0]['travel_time'] = a['travel_time']

        ox.plot_graph(G2, edge_color=ox.plot.get_edge_colors_by_attr(G2, attr='maxspeed'))

        fig, ax = ox.plot_graph(G2, edge_color=ox.plot.get_edge_colors_by_attr(G2, attr='travel_time'))

        edges_by_time = sorted(G2.edges(data=True), key=lambda e: e[2]['travel_time'], reverse=True)

        [e[2].get('name', 'Unnamed edge') for e in edges_by_time[:10]]

        nodes_centrality = nx.betweenness_centrality(G2, weight='length')

        for node, centrality in nodes_centrality.items():
        G2.nodes[node]['centrality'] = centrality

        ox.plot_graph(G2, node_color=ox.plot.get_node_colors_by_attr(G2, attr='centrality'))

    \end{minted}
\end{listing}

\clearpage

\begin{tcolorbox}[colback=gray!5!white, colframe=gray!60!black, title=Anexo D \- Fragmento de código en python para la generación del gráficas]
	En este anexo se presentan el ejemplo de un código utilizado para el análisis de datos generados por la simulación
    vehicular en SUMO y la generación de las gráficas presentadas en el capítulo.
	El código completo se encuentra disponible en el cuaderno ejecutable:
	\url{}.
\end{tcolorbox}

\begin{listing}[H]
    \caption{Fragmento de código para el análisis de datos de SUMO y generación de gráficas}
    \label{cod:anexo-sumo}
    \begin{minted}{python}
        # Cargar y obtener los promedios de cada archivo para todas las columnas
        datos_por_columna = {col: [] for col in cols}

        for archivo, etiqueta_archivo in archivos:
            df = pd.read_csv(archivo)
            for col in cols:
                promedio = df[col].mean()
                datos_por_columna[col].append((promedio, etiqueta_archivo))
    \end{minted}
\end{listing}