\chapter[Movilidad Inteligente]{Tecnologías para la movilidad inteligente}
\label{cp:movilidad-inteligente}

\insertminitoc
\parindent0pt

\section{Evolución de los sistemas de transporte inteligente (ITS)}
Los \acrfull{its} consiste en aplicar Tecnologías de la Información \acrfull{ti} incluyendo 
telecomunicaciones, sensores y automatización al sector 
del transporte y la movilidad. El objetivo principal es mejorar la eficiencia operativa, la seguridad 
vial, la gestión del tráfico y, fundamentalmente, optimizar la experiencia del usuario. Son un enfoque 
que busca mejorar la eficiencia del transporte a través de la aplicación de tecnología \cite{wootton_intelligent_1995}.
Los \acrshort{its} no representan una innovación repentina, sino el resultado de un proceso evolutivo y gradual. 
Su desarrollo ha sido impulsado por avances continuos en cuatro áreas tecnológicas clave: la mejora de 
las comunicaciones, que facilita el intercambio de datos en tiempo real; el progreso en la automatización 
de vehículos e infraestructuras; el crecimiento exponencial en la capacidad de procesamiento de 
información; y la creciente sofisticación de los modelos y arquitecturas de datos que optimizan la 
gestión de la movilidad urbana.

\subsection{Orígenes y primeros desarrollos}
En sus inicios, la gestión de tráfico se caracterizaba por sistemas de señalización estática. Estos 
operaban con tiempos de ciclo preprogramados o fijos, basándose estrictamente en horarios establecidos 
y modelos históricos, y sin capacidad de ajuste en tiempo real al volumen o a la demanda real del flujo 
vehicular.
Hay revisiones históricas que describen cómo los ITS emergieron gradualmente desde sistemas simples de 
semáforos y control básico.

\subsection{Comunicación Vehicular}
La comunicación entre los vehículos y con la infraestructura vial ha adquirido una importancia 
fundamental en los ITS más recientes, gracias a la implementación de las \acrfull{vanets}.  En las \acrshort{vanets}, los vehículos funcionan como puntos de conexión que están constantemente 
compartiendo y reuniendo datos. Esto permite ofrecer servicios de transporte novedosos, tales como la 
administración del tráfico, la asistencia en la navegación, la conducción sin intervención humana y la 
emisión de avisos o alarmas \cite{gillani_data_2022}.
La conectividad vehicular se define como la capacidad de un vehículo para establecer comunicación directa
con otros vehículos \acrfull{v2v} o con la infraestructura vial \acrfull{v2i}. Esta capacidad permite el intercambio
de datos en tiempo real. Si bien el objetivo final de esta tecnología es lograr una cobertura de red 
completa donde teóricamente "todos los vehículos que viajan por la carretera estén conectados", la 
conectividad es la habilidad intrínseca para iniciar y mantener dichas comunicaciones \cite{hussain_vehicular_2019}.
De igual forma, según \cite{al-heety_comprehensive_2020}, el cual describe cómo \acrfull{sdn} están 
siendo incorporadas para mejorar la gestión y coordinación de redes vehiculares, mejor seguridad y 
flexibilidad en \acrshort{its}. Las cuales se percibe como una solución moderna y fundamental para superar las 
limitaciones de las redes vehiculares heterogéneas, dinámicas y a gran escala como \acrshort{vanets} e \acrfull{iov}, al ofrecer control centralizado, programabilidad y eficiencia mejorada.

\begin{figure}[!htpb]
    \centering
    \begin{subfigure}{0.45\textwidth}
        \centering
        \includegraphics[width=\textwidth]{Figures/V2V.pdf}
        \caption{Tecnología \acrshort{v2v}}
        \label{fig:figure-v2v-04}
    \end{subfigure}
    \hspace{.5cm} % Adjust the space as needed.
    \begin{subfigure}{0.45\textwidth}
        \centering
        \includegraphics[width=\textwidth]{Figures/V2I.pdf}
        \caption{Tecnología \acrshort{v2i}}
        \label{fig:figure-v2i-04}
    \end{subfigure}
    \caption{Tecnologías de comunicación vehicular \acrshort{v2v} y \acrshort{v2i}.}
    \label{fig:figure-v2v-v2i-04}
\end{figure}

\subsection{Control Adaptativo y métodos basados en inteligencia artificial}
La creciente accesibilidad a los datos en tiempo real está impulsando la difusión de los sistemas 
adaptativos.  Estos autores proponen \cite{wang_adaptive_2024} una estrategia llamada SD3-Light con el fin de 
solucionar la congestión vehicular. Esta táctica se basa en el control en tiempo real de los semáforos, 
lo que implica ajustar dinámicamente sus fases y la duración de las mismas. Este trabajo de investigación 
utiliza específicamente el \acrfull{rl} como su enfoque principal.
Adicionalmente, el uso del aprendizaje profundo para pronosticar el flujo de vehículos hace posible 
prever y reaccionar a los cambios bruscos en las condiciones del tráfico. El autor \cite{qin_research_2023} propone 
un enfoque innovador para la \acrfull{tfp} urbano utilizando Big Data y \acrfull{dl}, con el objetivo de mejorar la precisión y apoyar el desarrollo 
urbano sostenible

\subsection{Tecnologías emergentes y comunicación no tradicional}
Los ITS más recientes utilizan el Big Data y modelos de predicción debido a la enorme cantidad de datos 
recopilados por sensores, dispositivos móviles y vehículos interconectados. Esto les permite prever dónde 
habrá atascos, mejorar las rutas y adaptar la planificación.
También, surgen tecnologías poco convencionales, un ejemplo es la que propone \cite{vieira_adaptive_2024} un sistema 
que emplea luces LED en vehículos e infraestructura (\acrshort{v2v}, \acrshort{v2i}) para la comunicación bidireccional, 
permitiendo la localización precisa de los vehículos y la recopilación de datos detallados sobre el 
tráfico, concluyendo su trabajo con que el control de tráfico adaptativo utilizando la tecnología \acrfull{v-vlc} 
es una tecnología prometedora para permitir una comunicación confiable y eficiente en entornos de 
vehículos conectados y tiene una promesa significativa para mejorar el flujo de tráfico


\section{ Herramientas tecnológicas aplicadas al control del tráfico}
Los ITS más recientes utilizan el Big Data y modelos de predicción debido a la enorme cantidad de 
datos recopilados por sensores, dispositivos móviles y vehículos interconectados. Esto les permite 
prever dónde habrá atascos, mejorar las rutas y adaptar la planificación.

\begin{itemize}
    \item \textbf{Sistemas de control adaptativo basados en sensores y localización de detectores:} 
    Instalar sistemas de control adaptativo es una de las tácticas más efectivas para mejorar la 
    administración del tráfico en los cruces.  Estos sistemas pueden alterar los tiempos de los 
    semáforos en tiempo real, basándose en los datos exactos que proveen los sensores de detección 
    ubicados estratégicamente.  De esta manera \cite{miletic_review_2022}, analiza el problema del tráfico denso 
    en las ciudades y sugiere que los \acrfull{atsc} son 
    la solución propuesta para enfrentarlo y concluye cómo es esencial el uso de algoritmos \acrfull{rl} para 
    lograr que los sistemas de semáforos puedan reaccionar de forma dinámica y se ajusten a los patrones 
    de tráfico en constante variación.
    Asimismo, \cite{zhu_adaptive_2019} propone un método de control de señales de tráfico adaptativo y de bajo costo 
    diseñado para reducir la congestión en las intersecciones, especialmente en regiones con presupuestos 
    limitados , logrando demostrar que el método propuesto logra una reducción significativa en el 
    retraso total del tráfico en intersecciones con ratios de volumen a capacidad tanto bajos como altos.
    \item \textbf{Algoritmos de aprendizaje automático / aprendizaje reforzado (Reinforcement Learning) 
    y control basado en datos históricos: } Retomando el punto anterior sobre RL, los algoritmos que 
    usan aprendizaje automático (particularmente aprendizaje reforzado) junto con características del 
    estado real del tráfico (colas, flujos, etc.) permiten optimizar secuencias semafóricas de forma 
    más eficiente frente a métodos estáticos o basados en reglas simples. En donde hablábamos sobre el 
    trabajo hecho por \cite{wang_adaptive_2024}.
    \item \textbf{Visión por computadora y sensores visuales para priorización de vehículos de emergencia:} 
    Una tecnología esencial para identificar a los vehículos de emergencia (camiones de bomberos o 
    ambulancias) y asignarles automáticamente preferencia en los semáforos es la visión artificial.  
    La finalidad principal de esta acción es disminuir notablemente los plazos de respuesta, así como, 
    disminuir el riesgo de que se pierdan vidas. Así como, \cite{nellore_traffic_2016} propone un sistema que utiliza 
    sensores visuales en las intersecciones para medir la distancia de los vehículos de emergencia y 
    contar el tráfico circundante, empleando la distancia euclidiana como la técnica más precisa. 
    Beneficiando de esta forma a la población dado el profundo y significativo impacto social que 
    generan los eventos de emergencia.
    \item \textbf{Frameworks y simulaciones para comparar controladores adaptativos:} el hecho de que 
    estas herramientas permiten experimentar con varias estrategias para gestionar el tráfico las hace 
    esenciales.  Lo más relevante es que, al evitar la necesidad de instalar estos sistemas complejos 
    físicamente en la infraestructura vial, los costes de implementación y prueba se reducen 
    considerablemente. Paralelamente, \cite{genders_open-source_2019} expone un marco de código abierto para la creación 
    y análisis de diferentes modelos de control adaptativo de semáforos en simulaciones. Este marco 
    aplica enfoques tradicionales para el control del tráfico, como los modelos de Webster, Max-pressure 
    y \acrfull{sotl}. Se ha logrado demostrar que estos marcos incrementan la 
    eficiencia en cuanto a tiempos de viaje y demoras.
    \item \textbf{Priorización de tráfico de emergencia y sistemas dinámicos:} además de la detección 
    visual, hay estudios que sugieren sistemas integrales que aseguran la prioridad de tránsito para los 
    vehículos de emergencia. Estos sistemas cumplen su propósito al modificar de manera dinámica las 
    señales de tráfico, llegando a ajustar incluso todo el ciclo semafórico de un cruce con base en 
    la detección en tiempo real de la cercanía de esos vehículos \cite{ariffin_real-time_2021}.
\end{itemize}
