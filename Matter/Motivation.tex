\ifthenelse{\equal{\LanguageOption}{spanish}}{%
    \chapter*{Motivación}
    La motivación fundamental para el desarrollo de la presente investigación surgió de la experiencia directa vivida dentro del entorno universitario, donde se evidenció la necesidad crítica de optimizar los procesos de gestión académica. Durante los últimos años de la carrera, se observó cómo la falta de una planificación basada en datos generaba una constante incertidumbre en la población estudiantil, afectando la toma de decisiones sobre qué asignaturas cursar y provocando, en múltiples ocasiones, la imposibilidad de inscribir clases esenciales para el avance del flujo curricular.

    Esta problemática se identificó no como un hecho aislado, sino como una realidad sistémica en la que el Censo Académico no cumplía su función predictiva, resultando en una oferta de horarios desconectada de la demanda real. Se experimentó la dificultad de acceder a cupos en asignaturas críticas y el impacto negativo que esto conllevó en el tiempo de graduación, lo que impulsó el interés por investigar cómo las herramientas tecnológicas modernas podrían mitigar estas ineficiencias y transformar la experiencia administrativa y educativa.

    En consecuencia, se planteó el desafío de trascender la simple observación del problema para proponer una solución técnica y rigurosa. Se buscó aplicar los conocimientos adquiridos en ingeniería de sistemas e inteligencia artificial para diseñar un modelo capaz de anticipar la demanda y optimizar la distribución de recursos. El objetivo se centró en demostrar que la implementación de algoritmos predictivos podía sustituir la intuición por la precisión matemática, reduciendo así la fricción que los estudiantes enfrentan periodo tras periodo al intentar armar sus horarios.

    Finalmente, este trabajo se concibió bajo el compromiso de aportar un legado funcional a la comunidad universitaria. La investigación se orientó a sentar las bases para una gestión académica más ágil y justa, donde la tecnología sirviera como un puente para garantizar que la oferta educativa respondiera verdaderamente a las necesidades de los estudiantes, promoviendo así un entorno más ordenado y propicio para el éxito profesional de las futuras generaciones.
}{%
    \chapter*{Motivation}
    \guideinfo{In the \textit{Motivation} section, express your gratitude to those who helped and supported your work. Start by thanking your advisors, mentors, or supervisors who provided guidance and expertise. Mention any colleagues, classmates, or team members who contributed to discussions or offered assistance. You can also acknowledge specific organisations, institutions, or funding sources that supported your research or work. Lastly, include any personal acknowledgments for family or friends who offered encouragement and moral support during the project. Keep this section sincere, concise, and professional.}
}

\MediaOptionLogicBlank